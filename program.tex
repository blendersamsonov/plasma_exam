\documentclass[10pt, a4paper]{article}
\usepackage[T2A]{fontenc}
\usepackage[english, russian]{babel}

\usepackage{graphicx}
\graphicspath{{./figs/}}

\usepackage{amsmath}
\usepackage{amssymb}
\usepackage{mathtools}
\usepackage{physics}
\usepackage{upgreek}
\usepackage{sectsty}
\usepackage{color,soul}
\usepackage[colorlinks=true,allcolors=blue]{hyperref}
\sectionfont{\centering}

\usepackage{geometry}
 \geometry{
 a4paper,
 margin=25mm,
 }

\title{Билеты к кандидатскому экзамену по физике плазмы}
\date{}

\begin{document}

\newpage
\section{Термодинамика плазмы}
\label{sec.1}

\subsection{Понятие плазмы, квазинейтральность, микрополя, дебаевский радиус, идеальная и неидеальная плазма.}
\label{sec.1.1}

Плазма~\cite{kotelnikov}. Ссылка на уравнение~\eqref{eq.Euler}, Рис.~\ref{fig.1.2.1}.

У плазмы вводят т.н. параметр неидеальности, который есть $g = \frac{1}{n * r_d^3}$(Kroll, eq. 1.3.1). Когда 
$g \ll 1$ говорят, что плазма идеальна, т.к. в этом случае её свойства, как газа, схожи с идеальным газом.



~\cite{kotelnikov}. Ссылка на уравнение~\eqref{eq.Euler}, Рис.~\ref{fig.1.2.1}.


\begin{equation}
    \label{eq.Euler}
    e^{i \pi} + 1 = 0
\end{equation}

\subsection{Условие термодинамического равновесия, термическая ионизация, формула Саха, корональное равновесие, снижение потенциала ионизации.}
\label{sec.1.2}

Картинка

\begin{figure}[h!]
    \center{\includegraphics[width=85mm]{test.pdf}}
    \caption{\label{fig.1.2.1} Подпись.}
\end{figure}

\subsection{Вырождение плазмы, статистика Больцмана и Ферми—Дирака, модель Томаса—Ферми.}


\section{Элементарные процессы}
\label{sec.2}

\subsection{Столкновения заряженных частиц, дальнодействие.}
\label{sec.2.1}

[Ю.П. Райзер, Физика Газового разряда, 3-е изд, стр. 29]

Из всех сил взаимодействия между атомными частицами медленнее вceгo спадают с расстоянием (как $1/r^{2}$)  кулоновские силы. Они обладают наибольшим дальнодействием.
За время пролёта t мимо иона, электрон отклоняется на угол $\theta$ . Его можно оценить как отношение полученной поперечной скорости к изначальной скорости:

Основную роль в рассеянии играют столкновения с большим прицельным параметром $\rho$ (рассеяние на малые углы),реализуются при  $\rho > r_{0}$- Кулоновского радиуса (радиус при котором кин. энергия электрона равна потенциальной $ mv^{2}/2 = e^{2}/ r_{0} $,$ r_{0}=e^{2}/mv^{2} $). С другой стороны, потенциал иона спадает $\sim exp(-r/d)/r $. То есть основной вклад вносят столкновения с прицельным параметром от  $r_{0}$   до $d$ (радиус дебая).
Поэтому полное сечение для кулоновских рассеяний $\sigma= \pi*r_{0}^{2} \int_{r_{0}}^{d} r_{0} d\rho/\rho =\pi*r_{0}^{2}*ln{d/r_{0}}$

$ln{d/r_{0}}$ - Кулоновский логарифм.

Столкновения атомных частиц могут иметь упругий и неупругий характер. При упругом соударении меняются направления движения партнеров, происходит обмен импульсом и кинетической энергией, но внутренние энергии и состояния частиц остаются неизменными.

\subsection{Частоты столкновений}
\label{sec.2.2}

[Ю.П. Райзер, Физика Газового разряда, 3-е изд, стр.20]

Число соударений определенного рода, которые данная частица (назовем: ее 1) в среднем совершает в 1 с, двигаясь в газе из частиц мишеней 2, называют частотой столкновений.

$\nu_{1}=N_{2}*v'*\sigma (v')$
, где $v'$ скорость сближения

Для газовой кинетики это тоже работает и если распределение частиц массы М по абсолютным скоростям описывается максвелловской функцией, то вместо скорости надо ставить среднее её значение $\bar{v'}=\sqrt{2} \bar{v}$ ; $\bar{v}=\sqrt{\frac{8kT}{\pi M}}$, где за массу надо брать приведённое её значение по 2-м частицам ($ M=\frac{M_{1} M_{2}}{(M_{1}+M_{2})} $), а за сечение рассеяния $\pi d_{mol}^{2} $

\subsection{Столкновения электронов с атомами (упругие и неупругие)}
\label{sec.2.3}


\subsubsection{Упругие}
\label{sec.2.3.1}
[Астапенко В.А.,Лисица В.С., столкновительные процессы в …... стр.30]


Вычисление сечения процесса является сложным кванто-механическим расчетом и зависит от скорости налетающего электрона. Транспортное сечение выражается сложным образом
\begin{equation}
\sigma_{tr}=4\pi (L^{2}+\frac{4}{5}\frac{\pi \alpha L}{a_{bor}}\frac{mV}{\hbar}+\frac{\pi^{2}}{6}\frac{\alpha^{2}}{a_{bor}^{2}}(\frac{mV}{\hbar})^{2})
\end{equation}
Здесь $L$ - длина рассеяния электрона на атоме. Первый член - короткодействующие силы. Последний - дальнодействующий потенциал деполяризационный. Промежуточный - интерфериционный от предыдущих двух (проявляются волновые свойства электрона), он может быть отрицательным (при отрицательном L).
Как именно происходит интерференция - см [И. мак-Даниэль, процессы столкновений в ионизированных газах,гл. 4, §4, стр 161] и [гл. 3, §15, пункт Д, “рассеяние  S волны на сферической потенциальной яме”]

\subsubsection{Неупругие}
\label{sec.2.3.2}

Ионизация

[Астапенко В.А.,Лисица В.С., столкновительные процессы в …... стр.37]

Так как после ионизации мы имеем ситуацию, что электрон улетает от иона, то это мы можем рассматривать как упругое рассеяние электрона на ионе, однако в сечение войдут не все электроны, а лишь с энергией выше, чем $E_{ion}$

$\frac{d\sigma^{(R)}}{d\Omega}=(\frac{Ze^{2}}{2mV^{2}sin^{2}(\frac{\theta}{2})})^{2}$

При рассеянии иону передается импулmc $\Delta p=2mVsin(\theta/2)$ и соответственно энергия $\Delta E=4 E sin^{2}(\theta/2)$ , поэтому можно перейти к интегрированию по энергиям.
$\delta \sigma =\frac {\pi e^{2}d\Delta E}{E(\Delta E)^2}$; 
$\sigma_{ion}=\int_{E_{ion}}^{E} d\sigma=\frac{\pi e^4}{E}(\frac{1}{E_{ion}}-\frac{1}{E})=\frac{\pi e^4}{E_{ion}^{2}}\frac{x-1}{x}$
, где $x=E-E_{ion}$ , то есть ионизация имеет пороговый характер


Возбуждение электронным ударом [Астапенко В.А.,Лисица В.С., столкновительные процессы в …... стр.50]


Уравнение рассматриваемого процесса имеет вид $e + A -> A* + e$. Согласно этому принципу атом при взаимодействии с электромагнитным полем ведёт себя как набор осцилляторов, которые ставятся в соответствие паре энергетических уровней $E_i$ , $E_j$ атомного спектра. Собственные частоты этих осцилляторов равны собственной частоте перехода $i -> j$ , $\omega _{i,j}=\frac{(E_j-E_i)}{\hbar}$ , а эффективность их взаимодействия с электромагнитным полем определяется силой осциллятора:
\begin{equation}
f_{i,j}=\frac{2m\omega_{ij} {|d_{ij}|}^{2} }{3\hbar e^{2} g_i}
\end{equation}

, где $g_i$ - статистический вес начального состояния. 

При кванто-механическом описании дипольный момент осциллятора перехода $d_{ij}$ представляет собой матричный элемент ооператора электрического дипольного момента между состояниями $|i>$, $|j>$. В случае возбуждения атома $i,j>0$ и $f_{i,j}>0$, для электронного перехода с уменьшением энергии $i,j<0$ и $f_{i,j}<0$. Также может быть $f_{i,j}=0$, тогда такие переходы называются дипольно-(или оптически) запрещенными. Если  $f_{i,j} \ne 0$ оптически-разрешенный переход.
Предполагая поле налетающего электрона в области локализации атома однородным, можно записать следующее уравнение для радиус-вектора осциллятора $rij$:
$r_{ij}''+\gamma_{ij}r_{ij}'+\omega_{ij}^{2}r_{ij}=f_{i,j}\frac{e}{m}E(t,\rho)$, где $\gamma_{ij}$-константа затухания,  $\rho$- прицельный параметр. Записывается скорость затухания в таком осцилляторе и ищется работа, которую совершает поле над осциллятором за всё время столкновения.
Вероятность возбудить атом будет $W_{ij}(\rho)=\frac {A_{ij}(\rho)}{\hbar_{ij}}$, полное сечение  будет $\sigma_{ij}=2\pi \int_{a}^{\inf} W_{ij}(\rho)\rho d\rho$. Тут, как и в кулоновском рассеянии основную роль играют рассеяние на малые углы, то есть с большим прицельным параметром.
\begin{equation}
\sigma_{ij}=\pi f_{ij} \frac {e^{2}}{m\Delta E_{ij} } \int_{a}^{\inf} |E(\omega_{ij},\rho)|^{2} \rho d\rho
\end{equation}
Далее - трудные выкладки с методом функции подобия. Самое главное, характер функции сигмы от энергии. Она выглядит схожим образом с зависимостью сечения ионизации. Также имеет пороговый характер, на бесконечности спадает как $ln(E)/E$. Имеет максимум в при $E=3,45*E_{ij}$
Для запрещенных переходов данный расчёт в дипольном приближении невозможен. ибо носит не дипольный характер. Дипольно-запрещённые переходы бывают двух типов: без изменения спина атома (дипольный момент перехода отсутствует из-за невыполнения правил отбора по орбитальному квантовому числу L, переход осуществляется за счёт прямого кулоновского взаимодействия) и переходы с изменением атомного спина (В этом случае возбуждение атома происходит за счет обменного взаимодействия между налетающим и атомным электроном).



\subsection{столкновения тяжелых частиц}
\label{sec.2.4}


[Астапенко В.А.,Лисица В.С., столкновительные процессы в …... стр.90]

Столкновения заряженных тяжелых частиц (ионов) с атомами и молекулами зависит, как и столкновения тяжелых нейтральных частиц, от поведения термов сталкивающихся партнеров. Эволюция во времени термов квазимолекулы, образованной сталкивающимися атомами, схематически представлена на рис. 
Переходы могут быть быстрыми (барновские) и медленные (адиабатические). В частности если будем рассматривать термы квазимолекулы, образованной сталкивающимися атомами. Зависимость потенциала от времени обусловлена их зависимостью траектории от времени.
Понятно, что с какой-то вероятностью электрон может перескочить с одного уровня на другой.

\begin{figure}[h!]
	\center{\includegraphics[width=70mm]{heavy_particle.JPG}}
	%\caption{\label{fig.2.4.1}}
\end{figure}



\subsection{Резонансная перезарядка}
\label{sec.2.5}

Резонансная перезарядка. Состоит в передаче электрона от одного ядра к другому в системе идентичных ядер: $A + A^{+} \rightarrow A^{+} + A$. Здесь, на первый взгляд, начальные и конечные состояния системы совпадают. Надо помнить, однако, что при этом происходит обмен скоростями между нейтральным атомом и ионом(направления движения каждого останется прежним)

Взаимодействие $V(R)$ в  определяющее переход электрона от одного ядра (атом) к другому (ион), определяется перекрытием волновых функций на двух ядрах, разделенных  
расстоянием $R$: $V(R)=exp(-\gamma R)$($\gamma=(2I)^{1/2}$ - определяется энергией связи $I$ электрона в атоме). Фактически при перезарядке электрон туннелирует между двумя потенциальными ямами.

\begin{figure}[h!]
	\center{\includegraphics[width=70mm]{res_recharge.JPG}}
	%\caption{\label{fig.2.4.2}}
\end{figure}
 Вероятность перехода:
\begin{equation}
w(\rho)=sin^{2}[ \int_{-\inf}^{+inf} V(R(t))dt] \sim sin^{2}[V(\rho)\rho/v]
\end{equation}
Как следует из формулы, вероятность перехода между двумя ядрами быстро осциллирует при малых относительных скоростях v в соответствии с многократным перескоком электрона от одного ядра к другому, в результате чего электрон в среднем с вероятностью 1/2  находится в потенциальной яме каждого из ядер. Интегрируя свесом $2\pi\rho d\rho$, получим сечение резонансной перезарядки:
\begin{equation}
\sigma_{res}\propto(frac{pi}{2\gamma^{2}})ln^{2}(v_0/v)
\end{equation}

, где $v_0$ - скорость электрона на атомной орбите (порядка атомной). Видно, что сечение резонансной перезарядки значительно превышает атомные сечения благодаря большому значению логарифма  отношения атомной скорости к скорости относительного движения ядер.  Например, это сечение равно $5*10^{-15} см^{2}$ для перезарядки протона на водороде при энергии 1 эВ.

P.S. есть ещё и нерезонансная перезарядка при нерезонансной перезарядке, соответствующей передаче электрона от одного ядра к другому в случае неидентичных ядер с разными энергиями 
связи электрона на каждом из них.

Особый случай соответствует перезарядке атома на многозарядном ионе, потенциал ионизации которого намного превосходит потенциал ионизации исходного атома. Здесь атомный электрон переходит в густой спектр высоковозбужденных состояний многозарядного иона, для которых энергия связи близка к энергии связи в исходном атоме.
\begin{figure}[h!]
	\center{\includegraphics[width=70mm]{res_recharge_1.JPG}}
	%\caption{\label{fig.2.4.2}}
\end{figure}

Наверное, из всего надо вынести то, что при малых скоростях/энергиях налетающего атома, сечение будет большим, чтобы электрон за большое время пролёта стуннелировал в другой атом/ион. Экспериментально это и было подтверждено.(см рис.) В случае сильно искривлённого поля иона, туда ему стунелировать гораздо проще, поэтому сечение перезарядки $\sim Zn^{4}$, где z - заряд иона, n- номер орбиты

\subsection{Рекомбинация}
\label{sec.2.6}
Рекомбинация [Астапенко В.А.,Лисица В.С., столкновительные процессы в …... стр.99]


Трехчастичная рекомбинация является процессом, детально обратным процессу ионизации и, следовательно, записывается уравнением $A^{+Z} + 2e -> A^{+(Z-1)} + e $.Суть процесса трехчастичной рекомбинации состоит в том, что электрон плазмы, взаимодействующий с ионом, отдает избыток своей энергии другому электрону плазмы и захватывается на  уровень иона. Ясно, что рассматриваемый процесс может играть доминирующую роль в плазме с достаточно высокой плотностью (чтобы присутствие третьей частицы было вероятным) и низкой температурой (чтобы радиус кулоновского взаимодействия между частицами был велик). Оценим скорость рекомбинации:
\begin{equation}
\alpha_3 n^2_e N_i = k_{step} N_{neutr} n_e
\end{equation}

Подставляя из Саха-больцмана значение для соотношения $\frac{N_e N_i}{N_{neut}}$ получаем оценку для скорости ступенчатой ионизации
\begin{equation}
	k_{step}=\frac{C}{(2\pi)^{3/2}}\frac{g_i}{g_a}\frac{me^{10}}{\hbar^{3}T^{3}}exp(-I/T)
\end{equation}

Для сравнения этой скорости со скоростью прямой ионизации ($k_{dir}$)(как бы ионизация напрямую электроном с энергией равной энергии ионизации):
 \begin{equation}
 	k_{dir}=v_{el} \sigma_{dir}=\sqrt{2T/m} a_{0}^{2} exp(-I/T)
 \end{equation}

 \begin{equation}
 		\frac{k_{dir}}{k_{step}} \approx (\frac{T}{I})^{\frac{7}{2}}
 \end{equation}


Видно, что в плотной низкотемпературной плазме процессы ступенчатой (каскадной) ионизации преобладают над процессами прямой ионизации. 

Диэлектронная рекомбинация (ДР) является наряду с трехчастичной и радиационной одним из основных процессов образования ионов меньшей кратности в плазме с достаточно низкой плотностью, где преобладают процессы парных соударений. Для реализации этого вида рекомбинации необходимо наличие у рекомбинирующего иона некоторого числа остаточных электронов, т.е. электронного остова. Действительно, для одновременного выполнения законов сохранения энергии и импульса в процессе рекомбинации необходимо присутствие третьего тела, роль которого в случае трехчастичной рекомбинации играет электрон плазмы, а в случае радиационной (фото-) рекомбинации - световой квант. Точно так же при ДР роль третьего тела выполняет электронный остов иона, берущий на себя избыток энергии рекомбинирующего электрона.  ДР преобладает над трехчастичной рекомбинацией в достаточно разреженной плазме (малые ne) и при достаточно высокой температуре плазмы.

\subsection{Диссоциативная рекомбинация}
\label{sec.2.7}
\begin{figure}[h!]
	\center{\includegraphics[width=70mm]{diss_recomb.JPG}}
	%\caption{\label{fig.2.4.2}}
\end{figure}
Первичным этапом диссоциативной рекомбинации является объединение молекулярного иона $АВ^{+}$ и электрона в квазимолекулу, которая находится в сверхвозбужденном автоионизационном состоянии. Соответствующая молекула $АВ$ может и не существовать, как в случае инертных газов, между атомами которых нет связи. Механизмом стабилизации захвата электрона ионом служит распад квазимолекулы на два атома; один из них может быть и возбужденным. Это возможно, если потенциальная кривая какого-либо из состояний системы атомов $А^{*} + В$ пересекает  потенциальную кривую иона $AB^{+}$. 
Системы $AB^{+} + e$ и $A^{*}+B$ при междуядерном расстоянии около $r_m$ имеют одинаковые энергии. Поэтому возможен самопроизвольный переход из первой конфигурации во вторую. 
Коэфициент диссоциативной рекомбинации равен:
\begin{equation}
\beta_{dis}=k_{capt}\frac{w_{stab}}{w_{stab}+w_{auto}}=\frac{k_{capt}}{w_{auto}}\frac{w_{stab}w_{auto}}{w_{stab}+w_{auto}}
\end{equation}
, где  $w_{auto} [c^{-1}]$- вероятность стабилизации (как и с излучением фотона - по факту обратное время жизни состояния) $w_{stab}\sim v_a[см/с] / a[см]$ скорость и размер частиц соответственно. Получается $w_{stab} \approx 10^{14} c^{-1}$, то есть это оч быстрый процесс. Он происходит даже быстрее, чем высвечивание кванта ($10^{8} c^{-1}$ )


\subsection{Прилипание.}
\label{sec.2.8}

 [райзер, стр. 145]
 
 
Процессы прилипания (attachment), сопровождающиеся образованием отрицательных ионов, важны для разрядов по двум причинам. Когда в тех газах, которые используются для исследований или приложений, имеются электроотрицательные компоненты, прилипание может играть заметную, иногда даже главную роль среди механизмов потерь электронов.
Они не зависят от механизмов образования и разрушения ионов и определяются уравнением Саха, в котором по смыслу роль положительных ионов играют нейтральные частицы, а роль атомов и молекул — отрицательные ионы:
\begin{equation}
	\frac{n_{-}}{n_e}=N\frac{g_{-}}{g_{a}}\frac{e^{I_{-}/kT}}{AT^{3/2}}
\end{equation}

\section{4. Динамика заряженных частиц в электрическом и магнитном полях}
\label{sec.4}

\subsection{Движение в скрещенных электрическом и магнитном полях.}
\label{sec.4.1}
(Ландау том 2(задачи) Котельник лекции)

Движение заряда в постоянном электрическом/магнтном или обоих полях: Ландау том2, параграфы 21-22(вроже это совсем 
примитивно, в конце есть задачи, хз будут ли это спрашивать).

\subsection{Дрейфовое приближение, разновидности дрейфового приближения.}
\label{sec.4.2}
Когда $\rho_H\ll l$, a $2\pi/\omega_B \ll \tau$, где $l$ - характерный масштаб изменения полей, a $\tau$ - характерное
время изменения полей, движение частицы можно разделить на быстрое вращение и медленное движение центра окружности.
Такое медленное движение называют дрейфом. Далее рассмотрим основные типы дрейфового движения.

\subsubsection{Электрический дреф}
\label{sec.4.2.1}
Запишем уравнение движения(пока для нерелятивизма):

\begin{equation}
    \begin{cases}
    \label{eq.4.2.1-motion}
        dr/dt = \upsilon \\
        dp/dt = e E + e/c [\upsilon \times B]
    \end{cases}
\end{equation}
легко видеть, что если перейти в систему, которая движется со скоростью $\upsilon_E = c\frac{E\times B}{B^2}$, то
уравнения движения примут вид:

\begin{equation}
    \begin{cases}
    \label{eq.4.2.2-motion}
    \dot{v_x}'=  \frac{eB}{mc} v'_y \\
    \dot{v_y}'= -\frac{eB}{mc} v'_x \\
    \dot{v_z}'=  \frac{e}{m} E_{\parallel}
    \end{cases}
\end{equation}
первые два уравенения дают движение по окружности, а третье - есть движение вдоль магнитного поля. В итоге в сопутствующей
системе отсчета (та, что со штрихом) имеем спираль. $\upsilon_E$ называют электрическим дрейфом, т.к. он вызван 
электрическим полем. По сути это дрейф нулевого порядка, тут поля постоянные и никаких неоднородностей нет. В плазме,
как правило, эта скорость мала по сравнению с тепловой, по этому этот дрейф отосят к первому порядку.
Также можно записать выражение для радиус-вектора ведущего центра (воображаемого центра дрейфующей системы). Есть два
способа это сделать:

\begin{equation}
    \label{eq.4.2.3-R1}
        \mathbf{R}'=\mathbf{r} + \frac{\mathbf{v}' \times \mathbf{B}}{\omega_H}
\end{equation}

\begin{equation}
    \label{eq.4.2.3-R2}
        \mathbf{R}=\mathbf{r} + \frac{\mathbf{v} \times \mathbf{B}}{\omega_H}
\end{equation}
В случае ~\ref{eq.4.2.3-R1} частица в сопутствующей с.о. заметает цилиндр (радиус ларморовской окружности постоянный), а
в случае ~\ref{eq.4.2.3-R2} будет гофрированная поверхность.

В релятивистском случае есть более общая офрмула для $\upsilon_E$, получаемая из преобразований Лоренца:

\begin{equation}
    \label{eq.4.2.4}
    \upsilon_E=c \frac{1-\sqrt{1-4\xi^2}}{2\xi^2} \frac{E \times B}{B^2 + E^2}
\end{equation}

где $\xi = |E\times B / (B^2 + E^2)|$. Легко видель, что, как с случае $E<B$, так и в обратном, это выражение дает 
скорость меньше скорости света. (???? $E=B$ ???).

\subsubsection{Дрейф под действем малой силы}
\label{sec.4.2.2}
В общем случае выражение для дрейфа получается из электрического заменой $e\mathbf{E}$ на $\mathbf{F}$:

\begin{equation}
    \label{eq.4.2.5}
    \upsilon_F=\frac{c}{e} \frac{\mathbf{F} \times \mathbf{B}}{B^2}
\end{equation}
также её можно получить усредняя уравнение движения $m\frac{d\mathbf{v}}{dt}=\mathbf{F} + \frac{e}{c} \mathbf{v}\times\mathbf{B}$
по периоду циклотронного вращения. Строго говоря, в уравнении~\ref{eq.4.2.5} стоит усредненная поперечная сила 
$\langle F_{\parallel} \rangle$, вила вдоль магниного поля дает нулевой вклад в дрейф. В случае, когда сила $F$ мала, есть
простой способ вычисления скорости дрейфа - надо усреднить по траектории(окружности), при остутствии силы $F$.

\subsubsection{Гравитационный дрейф}
\label{sec.4.2.3}

Пусть в качестве малой силы выступает гравитационная сила $m\mathbf{g}$. Скорость такого дрейфа 
$\upsilon_g=\frac{mc}{e} \frac{\mathbf{g}\times\mathbf{B}}{B^2}$ зависит от заряда частицы. Это вызывает ток в направлении $\upsilon_g$.

\subsubsection{Градиентный дрейф}
\label{sec.4.2.4}

Пусть электрического поля нет, а у магнитного есть градиент, перпендикутярный самому полю. Выполняя усреднение по
невозмущенной орбите легко получть: $\upsilon_{\nabla B}=\frac{v_{\perp}^2}{2\omega_B} \frac{\mathbf{B}\times\nabla_{perp}B }{B^2}$.

\subsubsection{Центробежный дрейф}
\label{sec.4.2.5}
В отстутствии внешней силы частица может иметь некоторую проекцию скорости на направление магнитного поля:
$\upsilon=v_{\parallel}  \frac{\mathbf{B}}{B}$, обозначим $\mathbf{h}= \mathbf{B}/B$. Если линии магнитного поля имеют некоторый
имеют некоторый изгиб, то на частицу будет действовать центробежная сила: $\mathbf{F}=-m\frac{d\upsilon}{dt}$. Распишем
производную от скорости:

\begin{equation}
    \label{eq.4.2.5-u}
    \frac{d\upsilon}{dt}=\frac{\partial \upsilon}{\partial t} + (\upsilon \nabla) \upsilon,
\end{equation}
где второе слагаемое есть движеие ведущего центра - это и есть центробежный дрейф, и выражение для скорости дрейфа:

\begin{equation}
    \upsilon_{cf} = -\frac{\mathbf{h}}{\omega_B} \times (\upsilon \nabla) \upsilon
\end{equation}
выражение для скорости можно упростить, если подставить выражения для скорости ведущего центра в нулевом порядке
и учесть, что в плазме $E/B \ll v/c$. Тогда выражение примет вид:

\begin{equation}
    \upsilon_{cf} = \frac{v_{\parallel}^2}{\omega_B} \mathbf{h} \times (\mathbf{h}\nabla)\mathbf{h},
\end{equation}
эта скорость направлена по бинормали магнитной линни и зависит от знака заряда.

\subsubsection{Поляризационный дрейф}
\label{sec.4.2.6}

Первое слагаемое в выражении~\ref{eq.4.2.5-u} отвечает за поляризоционый дрейф.

\begin{equation}
    \upsilon_{pol} = \frac{\mathbf{h}}{\omega_B} \times \frac{\partial \upsilon}{\partial t},
\end{equation}
далее подставляя выражение для скорости ведущего центра и учитывая, что электрическое поле близко к потенциальному
можно получить:

\begin{equation}
    \upsilon_{pol} = \frac{c}{B\omega_B} \frac{\partial \mathbf{E}_{\perp}}{\partial t},
\end{equation}
она зависит от заряда, по этому в переменном поле плазма поляризуется. Поскольку плазма это проводник, постояную
поляризацию разделеными зарядами создать нельзя, но когда есть переменное электрическое поле, возникает переменный 
поляризационный ток и он вызывает реальную поляризацию плазмы.

\subsubsection{Дрейф в неоднородном электрическом поле}
\label{sec.4.2.7}
Аналогично магнитному полу, получаем для скорости дрейфа:

\begin{equation}
    \upsilon_{\nabla E} = \frac{c\rho_B^2}{4 B^2} \nabla^2_{\perp}\mathbf{E}\times \mathbf{B},
\end{equation}
она зависит от заряда, и создает ток, это т.н. эффект конечного ларморовского радиуса.

\subsection{Заряженная частица в высокочастотном поле}
\label{sec.4.3}
(тут только нерелятовистский случай)

Пусть есть в.ч. поле: $\mathbf{E}=Re\mathbf{E}(\mathbf{r}\exp(-i\omega t), \mathbf{B}=Re\mathbf{B}(\mathbf{r}\exp(-i\omega t)$, где $\mathbf{E}(\mathbf{r}), \mathbf{B}(\mathbf{r})$ медленно меняющиеся амплитуды (по сравнению с характерым
ларморовским радиусом). Тогда движенеи разбивается на медленный дрейф и быстрые осцилляции: $\mathbf{r}=\mathbf{R}+\mathbf{\rho}$. 
Тогда разложим электрическое поле(магнитное убираем, т.к. нерелятивизм): $E(\mathbf{r})=E(\mathbf{R})+(\mathbf{\rho} \nabla)E(\mathbf{R})$. В нулевом приближении мы принебрегаем вторым членом и уравнение движения в первом порядке имеет 
следующий вид(? $\mathbf{R}$ убираем т.к. меняется слабо?):

\begin{equation}
    \label{eq.4.3-0}
    \ddot{\mathbf{\rho}}=\mathbf{E}(\mathbf{R}) \frac{e}{m} \exp(-i\omega t).
\end{equation}
В следующем поярдке усредним по периоду $2\pi/\omega$ радиус вектор ($\rho$ уйдет, т.к. в среднем 0):

\begin{equation}
    \label{eq.4.3-1}
    \ddot{\mathbf{R}}=-\frac{-e^2}{m^2 \omega^2}\langle \mathbf{E} \exp(-i\omega t) \nabla \mathbf{E} \exp(-i\omega t) \rangle, 
\end{equation}
подставляя сюда выражение~\ref{eq.4.3-0}, получаем выражение для пондермоторного силы:

\begin{equation}
    \label{eq.4.3pf}
    \ddot{\mathbf{R}}=-\frac{-e^2}{4 m^2 \omega^2} \nabla |E|^2.
\end{equation}
(что еще тут ?)

\subsection{Понятие адиабатического инварианта}
\label{sec.4.4}

\subsubsection{Понятие адиабатического инварианта}
\label{sec.4.4.1}
Площадь потока матнитного поля $\Upphi=\pi \rho B$ через ларморовскую окружность в сопутствующей системе является
адиабатическим инвариантом - приближенно сохраняется при медленном изменении полей. Постоянство следует из закона
Фарадея: $\mathcal{E}=-\dot{\Upphi}/c$, где по законе Ома $\mathcal{E}=IR=0$, т.к. сопративления нет т.к. принебрегаем
столкновениями.

Для случая частицы движузейся в однородных электрическом и магнитных полях, в общем случае адиабат. инвариант есть:

\begin{equation}
    \label{eq.4.3pf}
    J_{\perp} = \frac{1}{2\pi} \oint \mathbf{P}_{\perp} d\mathbf{r}_{\perp},
\end{equation}
где интеграл берется по пероду вращения частицы в с.с.о. Далее используя теорему Стокса можно получить, что:

\begin{equation}
    \label{eq.4.3pf}
    J_{\perp} = \frac{m\upsilon_{\perp}^2}{2\omega_B}=\frac{|e|}{2\pi c} \Upphi, 
\end{equation}
В общем случае адиабат. нвариантам являются следующие величины: $\Upphi$, $\gamma \mu$, $p_{\perp}^2/B$.

\subsubsection{Продольный адиабатический инвариант}
\label{sec.4.4.2}
В магнитных пробках, в нулевом приближении частица движется вдоль силовой линии от поворота к повороту. Интеграл вдоль
такого движения есть т.н. продольный адиабат. инвариант:

\begin{equation}
    \label{eq.4.3pf}
    J_{\parallel} = \frac{1}{2\pi}\oint\mathbf{P}_{\parallel}ds=1/\pi \int \sqrt{2m \left(\varepsilon - e\phi - \mu B - 0.5m\upsilon_E^2 \right)}. 
\end{equation}
Этот интеграл зависит от следующих параметров $J_{\parallel}=J_{\parallel}(x_0, y_0, \varepsilon, \mu)$, где $x_0, y_0$
есть координаты пересечения силовой линии с поверхностью рассекающую магнитную ловушку поперек.

\subsubsection{Третий адиабатический инвариант}
\label{sec.4.4.3}

В магнитной ловушке частица движется от одного конца к другому, смещаясь вдоль экватора, заметая некую поверхность - 
дрейфовую оболочку. Если магнитное поле и потенциал медленно меняются, сохраняется магнитный поток через сечение 
дрейфовой оболочки:

\begin{equation}
    \label{eq.4.3pf}
     \Upphi_{dr}= \int_{S_{dr}} \mathbf{B}d\mathbf{S}.
\end{equation}

Эти адиабат. инварианты удобны для нахождения завичимости энергии частицы от времени.


\section{ 7.Колебания и волны в плазме}
\label{sec.7}

\subsection{Основные типы колебания и волн в плазме: лэнгмюровские электонные и ионные, электромагнитные, ионно-звуковые, магнитозвуковые, альфвеновские.}
\label{sec.7.1}

В изотропной плазме легко записат дисперсионное уравнение:

\begin{equation}
    \label{eq.7.1}
    \left(1-\frac{\omega^2}{c^2 k^2} \varepsilon_{\perp}\right) E_{\perp} + \varepsilon_{\parallel} E_{\parallel}= 0
\end{equation}

Отсюда, при условии $\varepsilon_{\parallel} = 0$ получаем первый тип волн - продольные(ленгмюровские/электростатические) 
волны. Затухание Ландау(безстолкновительное затухание) относится именно к таким волнам. Их дисперсионка имеет следующий
вид:

\begin{equation}
    \label{eq.7.2}
    \omega^2=\omega_p^2 + 3/2 (k v_{t})^2
\end{equation}
волны с частотой сильно отличной от плазменное затухают.

В случае $1 - \frac{\omega^2}{c^2 k^2} \varepsilon_{\parallel}=0$ получаем поперечные волны(почти как в вакууме). Для них
дисперсионка:

\begin{equation}
    \label{eq.7.3}
    \omega^2=\omega_p^2 + (c k)^2
\end{equation}
Легко видеть, что волны с частотой $\omega < \omega_p$ не распространяются.

Если еще учесть ионы:

\begin{equation}
    \label{eq.7.4}
    \varepsilon=1+\frac{\omega_{pe}^2}{k^2 v_{Te}^2} - \frac{\omega_{pi}^2}{\omega^2}-3\frac{\omega_{pi}^2 k^2 v_{Ti}^2}{\omega^4}
\end{equation}
В случае низких частот(где влияние ионов существенно) можно принебреч 4ым слашаемым, тогда дисперсионка примет следующий вид:

\begin{equation}
    \label{eq.7.5}
    \omega=\frac{k c_s}{\sqrt{1 + k^2 r_d^2}}
\end{equation}
при малых $k$ имеем $\omega=k c_s$, где $c_s=\frac{T_i + T_e}{M}$ (ионный звук). При больших частотах имеем ионные ленгмюровские волны (сильно затухают, если температура ионов не мала). Грубо дисперсионные соотношения для разных волн изоражены на рисунке~\ref{plot.7.1}.

\begin{figure}[h!]
    \center{\includegraphics[width=85mm]{noB.pdf}}
    \caption{\label{plot.7.1} Дисперсионка для волн в плазме в отсутствии магнитного поля.}
\end{figure}

Магнитоактивная плазма

Пусть есть внешнее магнитное поле $B_0$, будем рассматривать холодную плазму. Вводя обозначения $n^2=\frac{c^2 k^2}{\omega^2}$, $\upsilon=\frac{\omega_p^2}{\omega^2}$, $u=\frac{\omega_H^2}{\omega^2}$, где $\omega_H$ - циклотронная частота,
дисперсионку можно записать следующим образом:

\begin{equation}
    \label{eq.7.6}
    n_{e,o}^2=1-\frac{2\upsilon (1-\upsilon)}{2(1-\upsilon) - u \sin^2(\theta) \pm \sqrt{u^2 \sin^4(\theta) + 4u(1-\upsilon)^2 \cos^2(\theta)}}, 
\end{equation}
где $\theta$ это угол между магнитным полем и волновым вектором. Индексы e,o соответствуют необыкновенной и обыкновенной волнам.
Cлучай $\theta=0$ представлен на графиках \ref{theta0-u-less-1, theta0-u-more-1}, важно обратить внимание на то, что формула~\ref{eq.7.6} не дает полного решения, нужно ещё учесть $\varepsilon_{\parallel}=0$.

\begin{figure}[h!]
    \center{\includegraphics[width=85mm]{theta0-u-less-1.pdf}}
    \caption{\label{theta0-u-less-1} Случай $\theta=0, u < 1$.}
\end{figure}

\begin{figure}[h!]
    \center{\includegraphics[width=85mm]{theta0-u-more-1.pdf}}
    \caption{\label{theta0-u-more-1} Случай $\theta=0, u > 1$.}
\end{figure}

Теперь расмотрим случай $\theta \ll 1$ (предельного перехода между $\theta=0$ и $\theta \neq 0$, при отсутствии
теплового движения и столкновений нет), дисперсионка в этом случае изображена на рис.~\ref{theta-small-u-less-1, theta-small-u-more-1}. При наличии необнородностей в областях сближения одна волна может переходить в другую.
\begin{figure}[h!]
    \center{\includegraphics[width=85mm]{theta-small-u-less-1.pdf}}
    \caption{\label{theta-small-u-less-1} Случай $\theta\ll1, u < 1$.}
\end{figure}

\begin{figure}[h!]
    \center{\includegraphics[width=85mm]{theta-small-u-more-1.pdf}}
    \caption{\label{theta-small-u-more-1} Случай $\theta\ll1, u < 1$.}
\end{figure}

Когда $\theta=\pi/2$ график дисперсионка имеет следующий вид~\ref{theta90}.
\begin{figure}[h!]
    \center{\includegraphics[width=85mm]{theta-90.pdf}}
    \caption{\label{theta-90} Случай $\theta=\pi/2$.}
\end{figure}

Альфвеновские волны. При рассмотрении предыдущих волн, движением ионов принебрегалось. Рассматриваем случай, когда волна распространяется вдоль вшеншего магнитного поля; также рассматриваются низкие частоты (сравнимые с циклотронной частотой ионов ?). Дисперсионка таких волн имеет следующий вид:

\begin{equation}
    \label{alfven}
    \frac{\omega}{k}=\frac{B_0}{\sqrt{4\pi \rho}}
\end{equation}
где $\rho$ есть масстовая плотнотсь плазмы (все частицы). Важно, что фазовая скорость не зависит от частоты! По сути это совместное движение/осцилляции ионов и магных силовых линий поперек движения волны. Они распространяются без дисперсии.

\subsection{Показатель преломления плазмы, пространственная и временная дисперсии, фазовая и групповая скорости плазменных волн.}
\label{sec.7.2}


Диэлектрическая проницаемость в холодной плазме с учетом ионов:

\begin{equation}
    \label{eq.7.7}
    \varepsilon=1 - \frac{\omega_{pe}^2}{\omega^2} - \frac{\omega_{pi}^2}{\omega^2},
\end{equation}
если есть столкновения (пусть отбросим ионы), то $\varepsilon=1-\frac{\omega_{pe}^2}{\omega (\omega+i \nu)}$. когда есть некоторый поток, движущийся со скоростью $\upsilon_0$, то без учета ионов и столкновений, имеем: $\varepsilon=1-\frac{\omega_{pe}^2}{(\omega - k \upsilon_0)^2}$.
Для затухания Ландау: $\varepsilon=1+\frac{4\pi e^2}{m} \int \frac{v}{\omega} \frac{\partial f_0}{\partial v} \frac{dv}{\omega-vk}$.

При наличии внешнего магнитного поля $B_0$ диэликтрическая проницаемость перестает быть изотропной(считаем, что трения нет и $\upsilon_T \ll \upsilon_{ph}$):

\begin{equation}
    \label{eq.7.8}
    \varepsilon=
    \begin{pmatrix}
        \varepsilon_B & ig & 0 \\
        -ig & \varepsilon_B & 0 \\
        0 & 0 & \varepsilon_{\parallel}
    \end{pmatrix}
\end{equation}
 где $\varepsilon_B=1-\frac{\omega_{pe}^2}{\omega^2-\omega_{Be}^2}-\frac{\omega_{pi}^2}{\omega^2-\omega_{Bi^2}}$ и $g=\frac{\omega_{pe}^2 \omega_{Be}}{\omega (\omega^2-\omega_{Be}^2} -\frac{\omega_{pe}^2 \omega_{Be}}{\omega (\omega^2-\omega_{Be}^2}$, $\varepsilon_{\parallel}=1-\frac{\omega_{pe}^2}{\omega^2}-\frac{\omega_{pe}^2}{\omega^2}$. 

Фазовые и групповые скорость изображениы на графиках~\ref{noB} и т.д. Из необычного - фазовая скорость всегда больше скорости света(но это не страшно). 

Пространственная и временная дисперсия(??? общие вещи):
В общем случаем имеем:
\begin{equation}
    \label{eq.7.8}
    \overrightarrow{D}=\iint \varepsilon(t,t',\overrightarrow{r},\overrightarrow{r}') \overrightarrow{E}(t',\overrightarrow{r}') d^3 r' dt,
\end{equation}
в случае однородной и стационарной среды имеем $\varepsilon(t-t', \overrightarrow{r}-\overrightarrow{r}')$, в этом случае
также справедливо $\varepsilon(\omega, \overrightarrow{k})= \varepsilon^*(-\omega, -\overrightarrow{k})$.

\section{Излучение плазмы}
\label{sec.10}
[Котельников, лекции по физике плазмы, глава 7]
В слабоионизированный плазме спектр излучения является линейчатым - за счет молекул, атомов и не полностью ионизованных
ионов. При увеличенни температуры, непрерывный спектр доминирует над дискретным. Полностью ионизованная плазма имеет
непрерывный спектр.
\subsection{Элементарные процессы, интенсивность спектральных линий, сплошные спектры, вынужденное испускание}
\label{sec.10.1}
Можно выделить два типа излучения с непрерывным спектром - тормозное и рекомбинационное; также есть циклотронное,
которое на низких частота имеет линейчатый спектр, а на высоких - становится непрерывм(синхротронное излучение). Однако,
в достаточо плотной плазме циклотронное излучение проглощается и не выходит из плазмы.

Излучение сопровождает переход атома/электрона/иона/молекулы из одного состояния в другое. Есть свободно-свободые переходы
- это есть тормозное излучение (столкновение электронов и ионов) и тормозное поглощение. Связанно-свободные переходы - 
процессы фотоионизации и фоторекомбинации (т.е. освобождение и связывание электрона соотв.). Обы этих типа дают 
непрерывный спектр.

Связанно-связанные переходы электрона между дискретными уровнями энергии приводят в линейчатому спектру. В молекулах также
бывают полосатые спектры - со же самое, но в молекулах бывает, что много уровней рядом образуют совего рода полосу.

\subsection{Пробеги излучения, перенос излучения в среде, оптически прозрачная и непрозрачная плазма, лучистая теплопроводность}
Пробеги излучения -?

\subsubsection{Перенос излучения в среде}
Распространение излученной электромагнитной энергии через некоторую среду может сопровождаться поглощением, излучением и рассеянием. Интенсивность определяется как энергия проходящая в ед. площали $dS$, в ед. $dt$ времени, в ед. $d\omega$ частот:

\begin{equation}
    \label{alfven}
    I_{\omega} = \frac{d\mathcal{E}_{\omega}(\mathbf{r},\mathbf{n},t)}{\cos(\theta)d\omega dSd\Omega dt},
\end{equation}
а уравнение переноса можно записать следующим образом:

\begin{equation}
    \label{alfven}
    \frac{1}{c} \frac{\partial I_{\omega}}{\partial t} + \mathcal{\Omega}\nabla I_{\omega} + (k_s + k_a) I = j_{\omega} + \frac{1}{4\pi} k_s \int I_{\omega} d\Omega,
\end{equation}
где $j_{\omega}$ - коэффициент излучения, $k_s, k_a$ - коэффициенты рассеяния и поглощения. Слагаемое с интегралом имеет смысл излучения рассеянного на поверхность извне.

\subsubsection{Лучистая теплопроводность}

Лучислая или нелинейная теплопроводность - суть в том, что коэффициент теплопроводности зависит от температуры, и уравнение теплопроводности становится нелинейным.


\section{Диагностика плазмы}
\label{sec.11}

Зондовые методы, оптические методы, СВЧ-методы, корпускулярные методы, лазерное рассеяние, магнитные измерения.

\subsection{Зондовые методы}
\label{11.1}

[И.М. Подгорный, лекции по диагностике плазмы, стр. 15]

\subsection{Оптические методы}
\label{11.2}

Определение электронной температуры по интенсивности излучения линейчатого спектра.

При релаксации возбуждённого состояния между уровнями испускается квант с частотой $\nu=\frac{E_2-E_1}{h}$ и интенсивность излучения из объёма будет равна $I=h\nu n_{2} A_{21}$, где $A_{21}$ - вероятность перехода (коэф. Эйнштейна).
\begin{equation}
	n_1 u(v)B_{12}+n_e n_1 <\sigma_1 v>=n_e n_2 (\sigma_2 v) + n_2 A_{21}+n_2 u(v) B_{21}
\end{equation}

, где $n_1$ - концентрация атомов в состоянии 1, $n_2$ - концентрация атомов в сост. 2, $n_e$ - концентрация электронов, $v$ - скорость электронов, $\sigma_1$ эффективное сечение возбуждения, $\sigma_2$ - эффективное сечение, снимающее возбуждение, $u(v)$ спектральная плотность излучения в плазме, $B_{12}, B_{21}$ - коэффициенты эйнштейна. В опытах обычно смотрят на излучение оптически прозрачной плазмы (ничего не поглощает) Есть 2 предельных случая.

\subsubsection{Плазма низкой концентрации}
\label{11.2.1}
Считаем так, если происходит мгновенное высвечивание (пренебрегаем снятие возбуждение ударами)
Уравнение выписывается упрощается до 
\begin{equation}
	n_e n_1 <\sigma_1 v>= n_k  \sum {i=1}{k-1}A_{k i}
\end{equation}
$A_{k i}$ коэффициент эйнштейна для релаксации с уровня k на уровень i.
Следовательно интенсивность излучения 
\begin{equation}
	I=h \nu n_k A_{k1}=h \nu n_e n_1 <\sigma_1 v>\frac{A_{k1}}{\sum{i=1}{k-1} A_{ki}}
\end{equation}

Для резонансной линии отношение $A/\sum A $ равно единице. так же, если мы знаем $n_e, n_i$ и зависимость $<\sigma_1 v>$ от температуры, то можем определить температуру плазмы по отношению относительных интенсивностей двух линий.

\begin{equation}
	\frac{I_1}{I_2}= \ frac {\nu_1 <\sigma_1 v>_1 A_1 \sum A_k^{(2)} }{\nu_2 <\sigma_2 v>_2 A_2 \sum A_k^{(1)} }
\end{equation}
Это самое общее выражение. Далее, можно делать различные предположения (наверное уже не нужно). например, аппроксимировать сечение возбуждения функциями (стр. 45-46).
Не обязательно даже смотреть сваливание на основной уровень.

\subsubsection{Плазма высокой концентрации.}
\label{11.2.2}

Это такая концентрация, что $n_e <\sigma_2 v> \gg A$ 
При выполнения этого условия в отсутствие фотовозбуждения устанавливается между детально обратными процессами $n_1 <\sigma_1 v> = n_2 <\sigma_2 v>$.
Используя известное выражение, для отношений вероятности прямого и обратного процессов:
\begin{equation}
	\frac{<\sigma_1 v>}{<\sigma_2 v>} = \frac{g_1}{g_2} e^{-\frac{E}{kT_e}}
\end{equation}
, где $E$ -  энергия уровня, $g_2$ и $g_1$  - статистические веса соответственно основного и возбуждённого состояний.
В этом случае значение интенсивности спектральной линии, отвечающий переходу из состояния с энергией возбуждения E в основное состояние равно:

\begin{equation}
	I=\frac{h \nu n_1 A g_2}{g_1} e^{-\frac{E}{kT_e}}
\end{equation}
Собственно, отношение двух линий для плазмы высокой концентрации, не зависит от концентрации, а зависит от температуру.
\begin{equation}
	\frac{I_1}{I_2}=\frac{ \nu_1 A_1 g_1}{ \nu_2 A_2 g_2} e^{-\frac{E_1 - E_2}{kT}}
\end{equation}

Т.е. если в эксперименте с электронами распределёнными по больцману измерить интенсивности линий, то можно вывести значение температуры плазмы. 

\subsubsection{определение параметров плазмы по форме спектральных линий}
\label{11.2.3}

Контуры спектральных линий атомов деформируются под действием внешних факторов.
\subsubsection{Определение ионной температуры по доплеровскому уширению спектральных линий}
\label{11.2.3.1}



Из-за допплера длина волны смещается на 
\begin{equation}
	\Delta \lambda = \lambda_0 (1+\frac{v_x}{c})
\end{equation}
Если распределение максвеловское, то  значение составляющей $v_x$  в пределах от  $v_x$  до $v_x + dv_x$ определяется формулой:
\begin{equation}
	\frac{dn}{n}=(\frac{M}{2\pi kT_i})^{1/2} e^{-\frac{Mv^{2}_x}{kT_i}} dv_x
\end{equation}

Из двух этих формул можно получить распределение спектральной плоскости.
\begin{equation}
	I=I_0 e^{-\frac{Mc^{2}}{2kT_i} (\frac{\Delta \lambda}{\lambda_0})^{2}}
\end{equation}

Обычно, определяют полуширину линии. Полуширина линии с температурой связана с температурой.
\begin{equation}
	\delta \lambda_D=\frac{2 \sqrt{2k ln(2)}}{c} \lambda_0 \sqrt{\frac{T_i}{M}}
\end{equation}


\begin{equation}
	T_i(eV)=1.7*10^{8}A(\frac{\delta \lambda_D}{\lambda_0})^{2}
\end{equation}
, где $A$- атомный вес иона.

\subsubsection{эффект доплера на рассеянном свете. Определение электронной температуры.}
\label{11.2.3.2}

Рассмотрим рассеяние монохроматического луча в плазме (оптически тонкой). Если длина волны падающего света много меньше среднего расстояния между заряженными частицами плазмы, а так же меньше дебаевского радиуса, то эффект взаимодействия излучения с плазмой определяется рассеянием на электроне формулой Томпсона: $\sigma =\frac{8\pi}{3} (\frac{e^{2}}{mc^{2}})^{2}$ и для электронов составляет $6.65*10^{-25} cm$.
При работе с плоскополяризованным светом рассеяние не является изотропным и мощность излучения под углом $\theta$ к направлению электрического вектора первичного излучения:

\begin{equation}
	W_{s} (\theta) d\theta=(\frac{e^{2}}{mc^{2}})^{2} W_0 \int {0}{L} n dl cos^{2}(\theta) d \theta
\end{equation}
, где n -  концентрация, а L - путь пучка в плазме.

При максвелловском распределении электронов по скоростям полуширина линии рассеяного света выражается следующим образом.

\begin{equation}
	\delta \lambda_s = 3.3*10^{-3} \lambda \sqrt{T_e (eV)}
\end{equation}
Это можно сделать в предположении не очень большой концентрации, когда лина волны лазера не слишком велика по сравнению с радиусом дебая.
\begin{equation}
	\alpha = \frac{\lambda}{4\pi sin(\frac{theta}{2} \lambda_D)} \ll 1
\end{equation}

, где $\theta$ - угол рассеяния, $lambda_D$ - дебаевский радиус

Это подробно рассмотрено в работе Розенблюта и Ростокера и можно получить окончательное значение для интенсивности света:
\begin{equation}
	I=I_0 exp(- \frac{1}{2} (\frac{\Delta \lambda}{\lambda} \frac{c}{\sqrt{kT_e /m}})^{2})
\end{equation}


\subsubsection{ Штарковское уширение спектральных линий в плазме. определение концентрации ионов.}
\label{11.2.3.3}

В плазме на размерах порядка радиуса дебая, в плазме присутствуют микрополя. Микрополя есть электронные и ионные. Электронные быстро осциллируют, а вот ионные - нет, которые в $\sqrt{M/m}$ раз меньше электронных. Скорости ионов в $\sqrt{m/M}$ раз меньше, то можно полагать, что неподвижный атом находится под быстрым воздействием электронов, находясь в слабоменяющимся поле иона. 

И энергитические уровни возбуждённого атома, находясь под действием этого внешнего электрического поля, расщепляются. А за счёт быстродействующих микрополей электронов, осуществляется переход между штарковскими подуровнями или в результате резкого изменения величины и направления поля происходит сильное изменение фазы излучаемой световой волны. 

Вызванное электронами уширение ОТДЕЛЬНОЙ штарковской компоненты спектральной линии может быть описанной с помощью ударной теории.
\begin{equation}
	I_s=\frac{\gamma}{(\omega + \omega_0 - d)^{2}+ \frac{\gamma^{2}}{4}}
\end{equation}
, где $\omega_0$ - частота линии, $d$ -  смещение линии в электрическом поле
По сути, полуширина линии - обратное время время жизни атома в данном возбуждённом состоянии $\gamma =n_e <\sigma v_e>$.
Сечение электронных столкновений можно записать как $\sigma \sim \pi \rho^{2}$

В квазиклассическом приближении переход из одного возбуждённого состояния в другое под действием столкновений возможен в том случае, есди время пролёта частицы становится сравним с временем перехода между состояниями. В водородном приближении это:
\begin{equation}
	\frac{v}{\rho}=\frac{<d_{m m’}> e}{\hbar \rho^2}
\end{equation}
Здесь $<d_{m m’}>$ - дипольный момент перехода штарковскими коспонентами. записывая его в виде $k^2 a_{0} e$.
Подставляя значение радиуса боровской орбиты приводит, к $\rho \sim \frac{k^2 \hbar}{Zmv}$. Поэтому подставляя это в формулу для сечения получаем, что:
\begin{equation}
	\sigma = \pi k^4 (\frac{h}{mv})^2 Z^2
\end{equation}

В действительности из-за дальнодействуюшего характера дипольного взаимодействия в правую часть этого выражения необходимо добавить множитель $L= ln(\frac{m v_e^{2}}{E-E_1})$ , котрый можно заменить 10.

Критерий применимости ударной теории - длительность столкновений много меньше времени между столкновениями, т.е. $\frac{n^{-1/3}}{v_e} << \frac{1}{n <\sigma v_e>} $ выполняется тем лучше, чем выше концентрация плазмы и чем выше температура.

Так же, данная теория хорошо описывает только хорошо отстоящие от центральной линии (при штарковском уширении происходит разложение несущей спектральной линии)

Для центральной линии это не выполняется (оч маленькие расстояния между подуровнями штарковскими $\omega - \omega_0 <= \gamma$).
Тут вступает в силу теория, которая затрагивает микрополя в плазме, а именно, Хольмарк выдвинул теорию о распределении микрополей в плазме, а именно вероятность поля:
\begin{equation}
	H( \beta ) = \frac{2}{\pi \beta} \int_{0}^{\inf} v sin(v e^{- (\frac{v}{\beta})^{3/2}}) dv
\end{equation}
, где $\beta = \frac{E}{E_0}$ , $E_0 = 2,61 e n^{2/3}$ 
Физический смысл $E_0$ прост: Среднее расположение между частицами плазмы $\frac{1}{n^{1/3}} $, а напряжённость поля точечного заряда обратно пропорциональна квадрату расстояния. Форма функции Хольцмарка:
(нужно картинку).
В этом случае контур линии центральной в рамках применимости запишется как 
\begin{equation}
	I_{st}=\frac{1}{\Delta \omega_0} H(\frac{}{\Delta \omega})
\end{equation}
В этой формуле в случае линейного эффекта Штарка: $\Delta \omega_0 = 2.61 \alpha n^{2/3}$

\subsubsection{Сплошной спектр. определение концентрации и электронной температуры}
\label{11.2.4}
Форма и интенсивность непрерывного спектра излучения плазмы определяется протеканием следующих процессов:
1.Тормозное излучение при взаимодействии электронов с ионами
2.Рекомбинационное излучение, возникающее при радиационном захвате электрона ионом.
3.Тормозное излучение, возникающее в результате электрон-электронных взаимодействий.

Из всех трёх остановимся на первых двух, ибо 3 играет роль в релятивистской плазме и обычно её не наблюдают.

Посмотрим на интенсивность рекомбинационных линий.
\begin{equation}
	I_rec d\nu = A n_i n_e (\frac{E_H}{kT_e})^{3/2} (\frac{E_{ik}}{E_H})^{2} \frac{g_{fb}}{k} \xi_k e^{\frac{(E_{ik}-h \nu)}{kT_e}} d\nu 
\end{equation}
Здесь $E_H = 13.6 eV$ энергия ионизации водорода. $E_{ik}$ - энергия ионизации оболочки, для которой главное квантовое число равно $k$, $g_{fb}$ - множитель Гаунта для свободно-связанных переходов.$ \xi_k$ - число квантовых мест на $k$-ой оболочке (если пустая, то $ \xi_k = 2k^{2}$ ). 
Видно, что мощность рекомбинационного излучения пропорциональна $Z^{4}/k^3$
Видно, что этот спектр имеет РЕЗКУЮ границу. Нету фотонов с энергией меньше чем $E_{ik}$, а с другой стороны есть длинный хвост, который описывается степенью экспоненты делённую на температуру. По этому хвосту и можно восстановить температуру

\begin{figure}[h!]
	\center{\includegraphics[width=70mm]{Intencity_recomb.JPG}}
	%\caption{\label{fig.2.4.1}}
\end{figure}

В случае тормозного излучения, обусловленного свободно-свободными переходами, в отличие от рекомбинационного, неограниченно простирается в сторону длинных волн. Мощность излучения в единице объёма плазмы в телесном угле $4 \pi$ , приходящийся в интервал частоты $d\nu$.

\begin{equation}
	I_torm d\nu = A Z^{2} n_i n_e (\frac{E_H}{kT_e})^{1/2} e^{-\frac{h \nu}{kT_e}} g_{ff} d\nu
\end{equation}
, где $g_{ff}$ множитель граута, который в широком диапазоне температур имеет сложный характер, но он порядка 1.

В общем, если смотреть на мощность излучения, то она для тормозного излучения будет:
\begin{equation}
	Q_torm=1.5 *10^{-25} Z^{2} n_e n_i T^{1/2}[eV]
\end{equation}
\begin{equation}
	Q_rec=5 *10^{-24} Z^{4} n_e n_i T^{-1/2}[eV]
\end{equation}

Видно, что для водородоподобной плазмы, они сравниваются для энергий 30eV. Свыше 100ev основной вклад в континуум даёт тормозное излучение.

\subsubsection{Сплошной спектр. ИК область}
\label{11.2.5}

Он интересен тем, что измерения проводятся в большом диапазоне энергий. Можно мерить и концентрацию и эл. температуру.
Необходимо выбрать такой участок спектра, внутри которого происходит переход от объёмного тормозного спектра излучения к поверхностному.
Поверхностное излучение отвечает случаю отвечает излучению чёрного тела и реализуется для слабого поглощения в плазме. Мощность излучения чёрного тела зависит только от температуры, то есть измерив его мощность - определим температуру.

По скольку в широкой области концентраций и электронных температур самопоглощение ИК излучение можно не учитывать, концентрация плазмы связана с температурой электронов формулой:
\begin{equation}
	n=3.9*10^{19} T_e^{\frac{1}{4}}[eV] (\frac{\int_{\Delta \nu g_{ff}} I(\nu) d\nu}{\Delta \nu g_{ff}})^{\frac{1}{2}}   cm^{-3}
\end{equation}

\subsubsection{Определение диэлектрической проницаемости плазмы. СВЧ метод}
\label{11.4}

Вспоминаем, что распространение ЭМ волн в плазме определяется диэлектрической проницаемостью.
\begin{equation}
	\epsilon = 1- \frac{\omega_{pl}^{2}}{\omega^{2}} \frac{1}{1-i \frac{\nu_{collision}}{\omega}}
\end{equation} 
, где $\nu_{collision}=\frac{2*10^{-5} n_i Z}{T_e^{3/2}[eV]}$, $\omega$ - частота волны

В случае малого поглощения, убирается мнимая часть.
\begin{equation}
	\epsilon = 1- \frac{\omega_{pl}^{2}}{\omega^{2}} 
\end{equation} 
Тоесть, если мы сможем измерить эпсилон для среды и конкретной частоты - узнаем и $\omega_pl = \sqrt{\frac{4 \pi e^{2} n_e}{m}}$, то узнаем концентрацию $n_e$

То есть что делают, берут какой-то СВЧ генератор, светят на плазму. Если проходит, значит концентрация меньше критической для данной частоты. Если отражается назад, то значит внутри есть слой с критической концентрацией. Это метод отсечки.
Оправданно это при условии $l \gg \frac{c}{\omega_pl}$
, где $l$ - толщина слоя плазмы с концентрацией выше критической, $\omega_pl$ - плазменная частота.

Если такой слой достаточно мал, то прибегают к интерферометрии. Собирают установку, где часть СВЧ сигнала проходит по участку, занятому плазмой, половина по свободному тракту. Далее, сигналы суммируют в интерферометре. По амплитуде суммарного сигнала, можно судить о разности фаз.
 
\begin{figure}[h!]
	\center{\includegraphics[width=70mm]{HF interferometre.JPG}}
	%\caption{\label{fig.2.4.1}}
\end{figure}
 
 Очевидно, что разность фаз - набег фазы по пути через плазму с коэффициентом преломления N.
 \begin{equation}
 	\phi = \frac{2 \pi}{\lambda} \int_{0}^{l} [N_0 - N(x)] dx 
 \end{equation} 
, здесь $l$ - размер плазмы, $N=\sqrt{\epsilon}$ - показатель преломления, $\lambda$ - длина волны СВЧ. Вообще, если тракт откачан воздухом, то $N_0 =1$


\subsubsection{Определение диэлектрической проницаемости плазмы. Применение лазеров}
\label{11.5}
Когда частоты СВЧ не хватает, для большой концентрации плазмы, используют лазеры.
Принцип абсолютно тот же, лазерный пучок пропускают через плазму и интерферируют самим с собой. На экране камеры получают интерфериционную картинку в виде полос. Потом, когда на пути пучка возникает плазма, в тех местах где увеличена концентрация, там уменьшается показатель преломления среды и как следствие, интерференционная картина превращается в кривые полосы. По кривизне полос в каждой точке, можно судить о набеге фаз.

\subsubsection{Корпускулярные методы диагностики плазмы}
\label{11.6}

\subsubsection{Использование пучков заряженных частиц}
\label{11.6.1}
С помощью пучка можно исследовать оч быстрые изменения плазмы. Конфигурации эксперимента оч разные, поэтому приводятся лишь некоторые из них.
Первое это зондирование пучком плазмы для определения радиальной и азимутальной компоненты поля в магнитной ловушке. Двигаясь продольно магнитному полю, при наличии поперечного поля электроны будут испытывать дрейф в скрещенных полях. Помещая электронную пушку в области одной пробки, люминисцентный экран в другой пробке, можем вижеть отклонение эл пучка на экране.
В отсутствие плазмы пучок двигается вдоль линий магнитного поля и делает точку на люминофоре. Наличие дрейфового движения приводит к смещение на экране на расстояние
\begin{equation}
	\Delta x=\frac{cE_y}{H_z} * \frac{l}{v}
\end{equation}
. где $l$ - длина пути пучка в плазме, $E_y$ - составляющая электрического поля, перпендикулярная магнитному полю, $v$ - скорость движения электронов вдоль линий поля.
Соответственно, смещение по радиусу пучка это азимутальная компонента поля, смещение по углу - радиальная компонента поля E.

Для определения потенциала плазмы относительно стенок используют метод, основанный на запирании пучка электронов. Время пролёта через плазму:
\begin{equation}
	\tau = \sqrt{\frac{M_i}{2Z_i}} \int_{0}^{l} \frac{dx}{\sqrt{V_0 +V(x)}}
\end{equation}
, где $Z_i$ - заряд иона.$M_i$ - масса иона, $l$ - размер плазмы, $V_0$ ускоряющий потенциал в источнике ионов, $V(x)$ - распределение потенциала вдоль пути пучка.
Если падение потенциала почти всё сосредоточено в дебаевском рабиусе, значительно меньшим $l$, то выражение будет
\begin{equation}
	\tau = \frac{l}{v_i}
\end{equation}
, где 
\begin{equation}
   v_i= sqrt{\frac{Z_l (V_0 + V_pl)}{M_i}}
\end{equation}

Область применимости- не должны влиять неустойчивости. Время прохождения через плазму должно быть меньше, чем время развития неустойчивостей.
\begin{equation}
	\tau_{neust} =(\frac{n}{n'})^{1/3} \frac{1}{\omega_{pl}}
\end{equation}
, где $n'$ - концентрация электронов в пучке.




\subsubsection{Измерение концентрации нейтральных и заряженных частиц.}
\label{11.6.2}

Прохождение пучка атомов или ионов через плазму сопровождается атомными и электронными столкновениями, приводящие к изменению заряда пучка. И то, на сколько он потеряет заряд - тоже будет функция температуры и концентрации электронов.
Рассмотрим изменение зарядового состояния монохроматического атомного пучка, состоящих из нейтральных атомов и протонов водорода, при прохождении через водородную плазму.
Средне квадратичный угол отклонения пучка будет:
\begin{equation}
 \overline{\theta} = 2 (\frac{Z_1 Z_2 e^{2}}{M v^2}) \sqrt{2 \pi nLl}
\end{equation}
, где $l$ - длина пути пучка в плазме с концентрацией $n$, $L$ - кулоновский логарифм ($L \sim 15$); $M$ - масса частицы пучка.
Для водорода с энергией электрона $10^{4} eV$ для $nl \sim 3*10^{16}$ среднеквадратичный угол будет $1.5^{\circ}$ 
Основные элементарные процессы плазмы с пучком.
1. Захват быстрым протоном электрона при взаимодействием с нейтральными атомами плазмы.
2. Обдирка электрона, принадлежащему быстрому атому, при взаимодействии его с нейтральными атомами среды.
3. Обдирка электрона, принадлежащему быстрому атому на протонахх плазмы.
4. Ионизация быстрых атомов электронами плазмы. 
Для описания каждого процесса, по факту пишут баланское уравнение по трассе пучка:
\begin{equation}
	\frac{dN_0}{dl} = -N_0 n_0 \sigma_{01}-N_0 n_i \sigma_{10} 
\end{equation}
и т.п., решаем с граничными условиями, сравниваем с экспериментом, делаем предположения о температуре плазмы и концентрации. Непосредственно сам пучок измеряется с помощью масс-спектрометров.


\subsubsection{Масс-спектроскопическая методика исследования быстрых частиц плазмы}
\label{11.6.3}

 Пучок, выходящий из плазмы, направляется в магнитное поле. В зависимости от массы и заряда, все частицы отклоняются на разные углы. Приближённо, отклонение на экране регистратора можно записать как
 
 \begin{equation}
 	x=\frac{ZeHl^{*}h}{Mv}
 \end{equation}
, где $l^{*}$ - расстояние между магнитом и детектором, $h$- протяжённость магнитного поля

\subsubsection{Калометрические измерение плазмы}
\label{11.6.4}
Есть несколько измерений.
1. Определение кривой распределения потоков тепла на стенку в относительных единицах.
2. определение энергии на данный участок поверхности.
3. Определение временного хода тепла.


Проще всего 1-я задача. Если на единице поверхности теплообменника выделяется количество тепла в Q калорий, то увеличесние его температуры определяется формулой.
 \begin{equation}
	\Delta \theta= \frac{Q}{dch} (1-\gamma)
\end{equation}
Здесь $d$ - плотность, $c$ - теплоёмкость и $h$ - толщина теплоприёмника, а $\gamma$ коэффициент отражения плазмы (обычно равен десятку процентов)
Вообще для измерения используют балометры. Так же можно исследовать мощность в зависимости от времени.
 \begin{equation}
	\Delta \theta= \int_{0}^{t} \frac{1}{dch} \frac{dQ}{dt}
\end{equation}

\subsubsection{Регистрация жёстких излучений плазмы}
\label{11.6.5}
Используют метод наведённой радиоактивности для того, чтобы обнаружить и измерить нейтронные импульсы интенсивности $10^4 - 10^5$ нейтронов и выше.
В естественном серебре есть два изотопа. $Ag^{107} Ag^{109}$ примерно в равных количествах. Сечения захвата 44 и 110 Барн соответственно. В следствие захвата образуются  $Ag^{108} Ag^{110}$ с периодом полураспада 2,4мин и 24,5сек, которые испытывают $\beta$ распад, который можно регистрировать. Собственно минимальная длительность определяется именно минимальным полураспадом, скважность между ними - максимальным временем полураспада.
Следующий метод - сцинтиляторный регистратор. ЧТо это такое. Ставят условный "друшлаг" - экран с большим количеством дырочек, за которым есть флуорисцирующий экран. Пучок нейтронов пролетает в дырки на друшлаге и попадают на экран, но из-за разлёта по углу, расстояние между светящимися точками на экране не такое как на друшлаге.

\begin{figure}[h!]
	\center{\includegraphics[width=70mm]{scintelator.JPG}}
	%\caption{\label{fig.2.4.1}}
\end{figure}

Так же используют Камеру обскура и Камеру Вильсона.


\section{ Гидродинамические и тепловые явления в плазме}
\label{13}

\subsection{Ударные волны в плазме}
\label{131}
[Зельдович Райзер, Физика ударных волн и ….. стр. 398]
Плазма характеризуется двумя температурами, электронной и ионной. Каждая может быть иногда описана максвеллом. Если меняется какой либо параметр, то характерное время максвеллизации составляет порядка времени столкновения $\tau_{collision}$. С температурами всё гораздо труднее, если одна из них изменилась, то вторая изменится за достаточно большое время, порядка $\tau_{ei}=\sqrt(\frac{M}{m}) \tau_{collision i}$ (передаётся малая часть энергии от электрона к иону). То есть, температура устанавливается не мгновенно.
Теперь рассмотрим ударную волну, летящую по ионизированному веществу.
На границе раздела двух плотностей и температур есть разрыв. Проходя по веществу, волна в области фронта разогревает ионы (как и нейтралы) в соответствие с показателем адиабаты $\gamma$ (для водородной плазмы 5/3). Так как в фронте создалось уплотнение ионов, они подтягивают за собой электроны, при чём это оч быстро происходит, так как плазма в целом квазинейтральна становится быть. Однако температура электронов из-за такого сжатия увеличивается всего лишь в $4^{\gamma - 1}=2,5$ раз. А выравнивания температуры с ионами не происходит, из-за медленности процесса, поэтому и могут реализоваться случаи, когда за границей разрыва температуры электронов и ионов сильно отличаются. Характерное расстояние, на котором они ещё разные $\Delta x = u_1*\tau_{ei} =u_1* \sqrt(\frac{M}{m}) \tau_{collision}$
\begin{figure}[h!]
	\center{\includegraphics[width=70mm]{13. Skach plotn.JPG}}
	%\caption{\label{fig.2.4.1}}
\end{figure}
так же, поскольку электроны являются быстрыми, то они хорошо переносят температуру. Коэффициент температуропроводности для них будут порядка: $\chi \sim l\bar{v}$ из-за сечений и масштаб длины порядка длин пробега $\lambda \sim \chi /D \sim l \bar{v}/D \sim l$
Коэффициент температуропроводности равен:
\begin{equation}
	\chi_e=\frac{l_e \bar{v_e}}{3} \approx \frac{\bar{v_e^{2}} \tau_e}{3}
\end{equation} 
где $l_e$ длина пробега электронов, $\bar{v_e}$ - их тепловая скорость, а $tau_e$ - время между столкновениями электронов друг с другом.
Как было показано ранее, длина пробега заряженых частиц не зависит от их массы, а зависит только от заряда и температуры $l \sim \frac{T^{2}}{Z^{4}}$. Для водородной плазмы (Z=1) длины пробега электронов одного порядка, однако, скорость электронов больше в $\sqrt{m_i / m_e}$ раз больше чем ионов. Поэтому и коэффициент теэлектронной теплопроводности и характерный масштаб, на которой разыгрывается процесс электронной теплопроводности:
\begin{equation}
	\lambda_e \sim \frac{\chi_e}{D} \sim \sqrt{\frac{m_i}{m_e}} \frac{\chi_i}{D} \sim \sqrt{\frac{m_i}{m_e}} l_i
\end{equation}
Этот маштаб такого же порядка, что и ширина релаксационной зоны выравнивания температур электронного и ионного газов:
\begin{equation}
	\Delta x \sim D \tau_{ei} \sim D \sqrt{\frac{m_i}{m_e}} \tau_i \sim \frac{D}{\bar{v}} \sqrt{\frac{m_i}{m_e}} l_i \sim \sqrt{\frac{m_i}{m_e}} l_i
\end{equation}

Поэтому по отношению к электронной теплопроводности градиенты в релаксационной зоне не малы и теплопроводностный теплообмен в этой зоне сравним с теплообменом между ионами и электронами. Электронная теплопроводность способствует скорейшему выравниванию температур за вязким скачком, так как она перекачивает тепло из более удаленных от скачка уплотнения слоев газа в передние, где электронная температура меньше. 
рассмотрим теперь на сколько глубоко проникает тепло перед скачком уплотнения.
Для этого запишем поток электронной тепропроводности:
\begin{equation}
	S=-\kappa_e \frac{dT_e}{dx}=-\chi_e c_e \frac{dT_e}{dx}
\end{equation}
, где $\kappa_e = \upchi_e c_e$ - коэффициент температуропроводности, $c_e$ теплоёмкость 1см3 электронного газа при постоянном объёме.
Вообще, эффективный коэффициент электронной теплопроводности равен:
$\kappa_e = \xi \frac{(kT_e)^(5/2) k}{m_e^{1/2} Z e^{4} ln(\Lambda)} = \xi *1.93*10^{-5}\frac{T^{5/2}}{Z ln|Lambda} $ эрг/сек*см*град
, где $ln\Lambda$ - кулоновский логарифм, а $\xi$ -число слабо зависящее от Z.
В силу стационарности процесса поток теплопроводности в прогревном слое равен гидродинамическому потоку электронной энергии.
\begin{equation}
	-S= D c_e T_e = \upchi_e c_e \frac{dT}{dx}
\end{equation}
(начальная температура электронов перед фронтом предполагается равной нулю; далеко перед волною поток S исчезает)
Замечая, что $\chi_e \sim \bar{v_e} \sim T_e^{5/2} $, или полагая $\chi_e=\alpha T_e^{5/2}$. Тогда интегрируем выражение и получим, что:
\begin{equation}
	x-x_0 = \frac{2}{5} \frac{a}{D} T_e^{-5/2}
\end{equation}
Или 
\begin{equation}
	T_e = [\frac{5}{2} [\frac{D}{a} (x-x_0) ]^{2/5}
\end{equation}
, где $x_0$ координата, где температура обращается в 0. Всё это изображено на рис ниже.

\begin{figure}[h!]
	\center{\includegraphics[width=70mm]{13. teploprov.JPG}}
	%\caption{\label{fig.2.4.1}}
\end{figure}

Скачок уплотнения [Зельдович Райзер, Физика ударных волн и ….. стр. 362]


Именно благодаря вязкости осуществляется необратимое превращение в тепло значительной части кинетической энергии газодинамического потока, набегающего на разрыв в системе координат, где разрыв покоится. 
Поскольку процесс ударного сжатия в скачке уплотнения разыгрывается на расстояниях, соизмеримых с газокинетическим пробегом молекул, при изучении структуры скачка следовало бы, строго говоря, исходить из представлений молекулярно-кинетической теории газов. 
Запишем уравнения одномерного течения вязкого и теплопроводного газа, стационарного в системе координат, связанной с фронтом волны: 

\begin{eqnarray}
	\frac{d}{dx} \rho u =0 \\
	2 \rho u \frac{du}{dx} + \frac{dp}{dx}-\frac{d}{dx} \frac{4}{3} \mu \frac{du}{dx} =0 \\
	 \rho uT \frac{d \sum}{dx}=\frac{4}{3} \mu (\frac{du}{dx})^{2} - \frac{dS}{dx}
\end{eqnarray}
, где $\sum$ - удельная энтропия, $\mu$  - коэффициент вязкости, $S$ - негидродинамический поток энергии, равный в случае теплопроводности $S=-\kappa \frac{dT}{dx}$ , где $\kappa$ - коэффициент теплопроводности.
Преобразуя третье уравнение G.1) с помощью второго закона термодинамики: 
\begin{equation}
	T \sum = d \epsilon + p dV = dw  - \frac{1}{\rho} dp 
\end{equation}
И интегрируя все уравнения получем первые интегралы системы:
\begin{eqnarray}
	\rho u = \rho_0 D \\  p+\rho u^{2} - \frac{4}{3} \mu \frac{du}{dx} = p_0 + \rho_0 D^{2} \\  w + \frac{u^{2}}{2} + \frac{1}{\rho_0 D} (S - \frac{4}{3} \mu u \frac{du}{dx}) = w_0 + \frac{D^{2}}{2}
\end{eqnarray}
Если подставить крайние значения и невозмущённые концентрации, то это будет прямо законы сохранения. 
Из этих соотношений следует, что скачок энтропии в ударной волне $\sum_1 - \sum_2 = \sum (p_1, \rho_1) - \sum (p_0, \rho_0)$  совершенно не зависит ни от механизма диссипации, ни от величины коэффициентов вязкости $\mu$ и теплопроводности $\kappa$. Они лишь определяют его толщину.
Вводится безразмерная величина $\eta = \frac{\rho_0}{\rho} $
И так же получается, что
\begin{equation}
	\frac{p}{p_0}=\frac{1 + \frac{M^{2}}{3} (1-\eta^{2})}{\eta}=\frac{4 \eta_1 - \eta^{2}}{(4 \eta_1 -1) \eta}
\end{equation}
где, $\eta_1 = \frac{1}{4}+\frac{5}{4} \frac{p_0}{\rho_0 D^{2}}=  \frac{1}{4} + \frac{3}{4} \frac{1}{M^{2}}$

где, $M=\frac{D}{c_0}$ - число Маха, $c_0$ - скорость звука в начальном состоянии 
Подставляя это всё в систему, получим дифференциальное уравнение:
\begin{equation}
	\frac{5}{3} \frac{\mu}{\rho_0 D} \eta \frac{d\eta}{dx}=-(1-\eta)(\eta - \eta_1)
\end{equation}
\begin{figure}[h!]
	\center{\includegraphics[width=70mm]{13. adiab udarn.JPG}}
	%\caption{\label{fig.2.4.1}}
\end{figure}
Оценим ширину фронта как:
\begin{equation}
	\delta = \frac{D-u_1}{(\frac{du}{dx})_{max}}
\end{equation}
По порядку она будет:
\begin{equation}
	\delta \sim l_0 \frac{M}{M^{2}-1}
\end{equation}
Применима до тех пор, пока $\delta$ больше $l$. например, $M=2$ то $\delta=3 l$.
Роли вязкости и теплопроводности в образовании скачка уплотнения 
Если не учитывать вязкости, то первые интегралы уравнений гидродинамики одномерного стационарного течения G.3) принимают вид: 

\begin{eqnarray}
	\rho u = \rho_0 D \\ p+\rho u^{2} - \frac{4}{3} \mu \frac{du}{dx} = p_0 + \rho_0 D^{2} \\ w + \frac{u^{2}}{2} + \frac{S}{\rho_0 D}  = w_0 + \frac{D^{2}}{2}
\end{eqnarray}
Из первых двух уравнений G.10) следует, что в процессе ударного сжатия в отсутствие Вязкости состояние частицы газа должно непрерывным образом меняться вдоль прямой на диаграмме давление — удельный объем. 
\begin{eqnarray}
	p=p_0 + \rho_0 D^{2} (1-\eta) \\ \eta=\frac{V}{V_0}
\end{eqnarray}
Это важное свойство течения невязкого газа иллюстрируется рис. 7.4, на котором изображена ударная адиабата и прямая, связывающая начальное и конечное состояния газа. Попытаемся решить систему уравнений: 

\begin{equation}
	\frac{T_1}{T_0}=1+ \frac{2 \gamma (\gamma-1)}{(\gamma+1)^{2}} (M^2-1)(1+\frac{1}{\gamma M^{2}})
\end{equation}


Релаксационный слой [Зельдович Райзер, Физика ударных волн и ….. стр. 362]
Для возбуждения некоторых степеней свободы газа *) часто требуется много соударений молекул, причем необходимые числа соударений, т. е. времена релаксации, для разных степеней свободы могут сильно различаться. 
Время установления полного термодинамического равновесия во фронте ударной волны, а следовательно, и ширина фронта определяются наиболее медленным из релаксационных процессов. При этом, конечно, следует принимать во внимание только те процессы, которые приводят к возбуждению степеней свободы, дающих заметный вклад в теплоемкость 
при конечных параметрах газа. Если тшах — наибольшее время релаксации, а щ — скорость движения газа за фронтом относительно самого фронта, то ширина фронта порядка $\Delta x \sim u_1 \tau_max =D(\rho_0 / \rho_1) \tau_max$.

При комнатных температурах вращения в молекулах возбуждаются также быстро, в результате небольшого числа соударений; колебания же при этих температурах обычно не играют роли. Следовательно, ширина фронта слабых ударных волн, распространяющихся по молекулярному газу, нагретому до комнатной температуры, порядка нескольких газо-кинетических пробегов. 

При температурах порядка 1000° К, когда величина кТ сравнима с энергией колебательных квантов молекул $h\nu_osc$, возбуждение колебаний требует многих тысяч, а иногда десятков и сотен тысяч соударений. Ширина фронта ударной волны соответствующей амплитуды определяется временем релаксации для колебательных степеней свободы. 
Диссипативные процессы — вязкость и теплопроводность — играют роль только в области больших градиентов гидродинамических величин, т. е. в зоне, где возбуждаются быстро релаксирующие степени свободы. Эта зона в какой-то мере совпадает с областью вязкого скачка уплотнения. В зоне медленной релаксации, растянутой на расстояния многих газокинетических пробегов, градиенты малы и диссипацией можно пренебречь.
\begin{figure}[h]
	\centering
	{\includegraphics[width=50mm]{13. 2 udern adiab.JPG}}
	%\caption{\label{fig.2.4.1}}
	
	{\includegraphics[width=50mm]{13.rel sloy.JPG}}
	%\caption{\label{fig.2.4.1}}
	
\end{figure}
В данном случае, у нас есть 2 адиабаты. Адиабата / проходит круче, чем II, как показано на рис. 7.12. Действительно, при одинаковой плотности температура и давление газа при 
условии замороженности некоторых степеней свободы выше, так как, грубо говоря, одинаковая энергия сжатия распределяется по меньшему числу степеней свободы.
В случае, если волна настолько слаба, что скорость ее меньше скорости звука, соответствующей замороженности части степеней свободы, прямая АС проходит ниже касательной к адиабате / в точке А (рис. 7.13). При этом состояние непрерывным образом, меняется вдоль прямой АС от точки А до точки С, и в газе с самого начала происходит постепенное возбуждение замедленной части теплоемкости. 

Поскольку в релаксационной зоне удельная энтальпия почти неизменна, а теплоемкость по мере возбуждения ранее замороженных степеней свободы возрастает, температура в ней уменьшается. Уменьшение температуры может быть довольно значительным, если запаздывающая часть теплоемкости велика и вносит большой вклад в конечную теплоёмкость газа. Конечная температура $Т_1$ может быть в два-три раза меньше температуры Т' за скачком уплотнения. Точно так же значительно может возрастать и плотность газа (грубо говоря, $р \sim \rho T$; р меняется мало, а Т сильно). Профили р, q, и, Т во фронте ударной волны, распространяющейся по газу с замедленным возвозбуждением части теплоемкости, изображены схематически на рис. ниже. 




\section{Прикладные проблемы физики плазмы}
\label{sec.14}


\subsection{Управляемый термоядерный синтез}
\label{14}

[Кенро Миямото, Основы Физики Плазмы и управляемого термоядерного синтеза, Москва, физматлит 2007]

Всем известно про дефект масс, что при слиянии ядер до Fe и Ni выделяется энергия.
основная реакция D+T-> 4He + n [17.59 MeV], 1/5 энергии приходится на He и 4/5 на n
Чтобы сблизить ядра на достаточное расстояние надо преодолеть кулоновский барьер. ту силу, с которой они отталкиваются ($Z_{alpha}Z_p e^2/r_N$) вычислили, что примерное расстояние $r_{n}=1.4*10^{-13}*(A^{1/3}_{\alpha}+A^{1/3}_{\beta}) [cm]$ Но это оказалось слишком мало, ибо не сходилось с вычислениями T солнца, для 10кэВ вычислили точку остановки $R_t=10^{-11} [cm]$

[Миямото стр. 23]

Почему вообще надо удерживать плазму, какие характерные времена? Критерий Лоунсена 
Дело обстоит в том, что система должна сама себя поддерживать и при этом отдавать эл. энергию. Энергия в единице объёма $(3/2)nk(T_i+T_e)$. Время остывания $\tau_E = \frac { (3/2)nk(T_i+T_e) }{P_L+R}$, , где $P_L$ и $R$ соответственно мощность потерь энергии на диффузию и излучение плазмой. В то же время, в объёме происходит выделение тепла за счёт термоядерного синтеза мощностью. $P_{NF}=(n/2)(n/2)<\sigma v> Q_{NF} $. Потому
 как D и T присутствуют в реакторе в равном количестве и концентрация каждого n/2 соответственно. $<\sigma v> $ - вероятность процесса,   $Q_{NF} $ энергия на одну реакцию.
Чтобы всё было устойчиво и не затухало: $P_{heat}=P_L + R = \frac{3nkT}{\tau_E} < \eta_{el} \eta_{heat} P_{NF} $, где $\eta_{el} $ - КПД получения эл. мощности ,$ \eta_{heat} $ - КПД нагрева.

Итого: $ \frac{3nkT}{\tau_E} < \eta_{el} \eta_{heat} Q_{NF} n^{2} \frac{<\sigma v>}{4}$

\begin{equation}
	\label{eq.Disp14.1.1}
	n \tau_E > \frac{12 kT}{\eta Q_{NF} <\sigma v>} > 1.7*10^{20} m^{-3} s
\end{equation}
 - Критерий Лоунсена , критерий термоядерности. То есть надо или увеличивать концентрацию, или время удержания. Поэтому есть как токамаки с непрерывным временем работы, так и импульсные(импульсно поддерживается магнитное поле).
Понятно, что чем горячее плазма, тем труднее её удерживать, понятно что удерживать надо именно магнитным полем.

\subsection{Магнитное удержание}
\label{14.1}

Уравнения магнитной гидродинамики тут написаны [Голант Жилинский Сахаров, Основы ФП,  §10.1]

Из уравнения для средней скорости $\rho (du/dt)=(1/c)[j H]-grad\;p$; где $p=p_e+p_i$ - суммарное давление, $j=en(u_i+u_e)$ - плотность тока, $ du/dt=\delta u / \delta t + (u grad)u$ полная производная

можно вывести условие равновесия
\begin{equation}
	\label{eq.Disp14.1.2}
	\frac{1}{c}[j H]=grad\;p
\end{equation}
[Голант Жилинский Сахаров, Основы ФП,  §10.2]


Идея состоит в том, что преодолеть давление плазмы как газа с ооочень большой температурой с помощью давления магнитного поля около стенок.
Первая задача, возникающая при рассмотрении удержания плазмы в магнитном поле, заключается в определении условий, при которых достигается равновесие, т. е. электродинамические силы, действующие на каждый элемент объема плазмы, уравновешивают градиент давления.
\begin{equation}
	\label{eq.Disp14.1.3}
	G_H=\frac{1}{c} [j H]=\frac{1}{4 \pi} [H rotH]=-\frac{1}{8 \pi} grad(H^{2})+\frac{1}{4 \pi} (H grad) H
\end{equation}
Так же есть отдельный случай, когда линии имеют радиус кривизны R.

\begin{equation}
	\label{eq.Disp14.1.4}
(\vec H grad)\vec H=H(\vec h grad)\vec h H=H^{2} grad_{||}\vec h + \frac{1}{2} grad_{||} H^{2}=-H^{2} \frac{\vec R}{R}+\frac{1}{2} grad_{||} H^{2}
\end{equation}


, где $\vec h= \frac{\vec H}{H}$; $grad_{||}=(\vec h \; grad)$ - проекция градиента на направление магнитного поля. Подставляем в предыдущее выражение и получаем, что давление:
\begin{equation}
	\label{eq.Disp14.1.5}
 G_H=-grad_{\perp}(H^{2}/8\pi)-(H^{2}/4\pi R^{2})\vec R
\end{equation}
Первое слагаемое в представляет собой поперечный градиент введённого магнитного давления. Действие этой силы можно описать как взаимное «расталкивание» силовых линий в поперечном направлении. Второе слагаемое определяет силу, направленную к центру кривизны силовых линий, и называется натяжением магнитного поля. Оно формально получается, если приписать силовым линиям свойства растянутой струны.

Для равновесия: 
\begin{equation}
	\label{eq.Disp14.1.6}
 grad p + grad_{\perp}(H^{2}/8\pi)+(H^{2}/4\pi R^{2})\vec R =0
\end{equation}

Для случая, когда силовые линии прямые ($R\rightarrow \inf $), уравнение сводится к постоянству суммы кинетического и магнитного давлений в плоскости, перпендикулярной магнитному полю:
\begin{equation}
	\label{eq.Disp14.1.7}
p + H^{2}/8\pi=const
\end{equation}
\begin{equation}
	\label{eq.Disp14.1.8}
p_0 + H_{0}^{2}/8\pi=H_{e}^{2}/8\pi
\end{equation}

Равенство показывает, что максимальное давление плазмы, которое можно удерживать магнитным полем, равно магнитному давлению вне плазмы $H_{e}^{2}/8\pi$. При описании магнитного удержания часто вводят коэффициент $\beta$, представляющий собой
отношение давления удерживаемой плазмы к максимально возможному: $\beta=p/p_{max}=8\pi p/H_{e}^{2}$
Но это удержание поперёк линий магнитного поля, а строить комплексы в виде торов не всегда удобно, поэтому было придумано пробочное удержание.

пробочное удержание [Миямото стр. 32]

Понятно, что всё движение электрона в магнитном поле можно описать как движение вдоль и поперёк поля. Вдоль - свободное движение, поперёк - ларморовское вращение.
Если рассмотрим электрон в поле, он будет вращаться по кругу с радиусом равному Ларморовскому. Его можно представить в виде витка с током. У витка с током есть магнитный момент.
\begin{equation}
	\label{eq.Disp14.1.9}
 \mu=I*S=\frac{q\Omega}{2\pi} * \pi \frac{\rho^{2}}{2}=\frac{mv_{perp}^{2}}{2B}
\end{equation}	
Но кин энергия должна сохранятся при движении в магнитном поле, поэтому электрон будет лететь вдоль поля пока
\begin{equation}
	\label{eq.Disp14.1.10}
	\frac{mv_{||}^{2}}{2}+\frac{mv_{\perp}^{2}}{2}=\frac{mv^{2}}{2}=const
\end{equation}	
Поскольку магнитный момент сохраняется, то:
\begin{equation}
	\label{eq.Disp14.1.11}
 v_{||}=+-(\frac{2}{m}E-v_{\perp}^{2})^{1/2}=+-(v^{2}-\frac{2}{m} \mu B)^{1/2}
\end{equation}	
То есть электрон может полностью убрать продольную скорость и отразиться. Почему возникает этот эффект ? Потому что на магнитный диполь действует сила $-\mu \grad_{||}$ B. Ну и вводится пробочное отношение - отношение магнитных полей в центре и на краях ловушки $R_M=\frac{B_M}{B_0}$


\subsection{Плазменные источники излучения}
\label{14.4} 
раз­дел фи­зи­ки плаз­мы, изу­чаю­щий кол­лек­тив­ные взаи­мо­дей­ст­вия плот­ных по­то­ков (пуч­ков) за­ря­жен­ных час­тиц с плаз­мой и га­зом, при­во­дя­щие к воз­бу­ж­де­нию в сис­те­ме ли­ней­ных и не­ли­ней­ных элек­тро­маг­нит­ных волн и ко­ле­ба­ний, и ис­поль­зо­ва­ние эф­фек­тов та­ко­го взаи­мо­дей­ст­вия.
плаз­мен­ные ус­ко­ри­те­ли, ос­но­ван­ные на яв­ле­нии кол­лек­тив­но­го ус­ко­ре­ния тя­жё­лых за­ря­жен­ных час­тиц элек­трон­ны­ми пуч­ка­ми и вол­на­ми в плаз­ме

плаз­мен­но-пуч­ко­вый раз­ряд, ос­но­ван­ный на кол­лек­тив­ном ме­ха­низ­ме взаи­мо­дей­ст­вия плот­ных пуч­ков за­ря­жен­ных час­тиц с га­зом;

тур­бу­лент­ный на­грев плаз­мы плот­ны­ми пуч­ка­ми за­ря­жен­ных час­тиц и кол­лек­тив­ные про­цес­сы при транс­пор­ти­ров­ке и фо­ку­си­ров­ке пуч­ков в про­бле­ме управ­ляе­мо­го тер­мо­ядер­но­го син­те­за (УТС);

[Богданкевич Л С, Кузелев М В, Рухадзе А А "Плазменная СВЧ электроника" УФН 133 3–32 (1981)]
DOI: 10.3367/UFNr.0133.198101a.0003
URL: https://ufn.ru/ru/articles/1981/1/a/

Когда говорят о сильноточных электронных пучках, то прежде всего имеют в виду пучки с током, превышающим так называемый предельный вакуумный ток. Известно, что в металлическом волноводе с радиусом R и длиной $L >> R$, который обычно используется в вакуумной электронике в качестве резонатора, ток пучка ограничен пространственным зарядом электронов, причем предельный ток по порядку величины определяется соотношением:
\begin{equation}
	\omega_{b \; vol}^{2} \approx \frac{c^{2}}{S} \gamma
\end{equation}
, где $\gamma = [1-(u^{2}/c^{2})]^{-1/2}$ - — релятивистский фактор энергии электронов, $S$ — поперечное сечение пучка, меньшее или порядка сечения волновода, а $\omega_{b}=\sqrt{4\pi e^{2} n_{b} /m}$ — лэнгмюровская частота электронов пучка. 


Следовательно, сильноточным следует считать пучок, в котором $\omega_{b}^{2}>\omega_{b\;vol}^{2}$. Такой пучок может распространяться в волноводе только при наличии нейтрализации пространственного заряда электронов, что достигается заполнением системы относительно плотной (по сравнению с плотностью пучка) плазмой. Таким образом, сильноточная СВЧ электроника, строго говоря, может быть только плазменной. При этом, однако, плазма может и не влиять существенным образом на частоты генерируемых пучком электромагнитных волн. Дело в том, что длины волн возбуждаемых пучком электромагнитных колебаний меньше или порядка поперечных размеров электродинамической системы генератора, т. е. резонатора, точнее $\omega>\omega_{crit}= \mu c/R$, где $\mu$ характеризует радиальное волновое число моды колебаний (корни функций Бесселя или ее производных); обычно $\mu \sim 3-10$. Если плазма, заполняющая резонатор, имеет относительно низкую плотность, так что $\omega_{pl}=\sqrt{4\pi e^{2} n_{p} /m} < \mu c/R$ , то она существенно не меняет электродинамику резонатора, который по своим электродинамическим свойствам остается практически вакуумным. Вместе с тем такая плазма может нейтрализовать пространственный заряд пучка и позволить пропустить через резонатор сильноточный пучок со сверхпредельным током. Однако ток пучка при этом ненамного может превышать предельный, не более, чем в $\mu^{2} S/\gamma R^{2}$ раз, что следует из неравенств $\omega_{b}^{2}<\omega_{p}^{2}<\mu^{2} c^{2}/R^{2}$.

Иное положение имеет место в случае достаточно плотной плазмы, когда $\omega_{pl}>\mu c/R$. Роль такой плазмы в резонаторе не ограничивается нейтрализацией пространственного заряда пучка, она существенно меняет всю электродинамику резонатора и, в частности, спектры частот собственных электромагнитных мод резонатора. Именно в этом случае мы имеем дело с настоящей плазменной электроникой, которая, как будет видно из дальнейшего, позволяет продвинуться в область токов электронного пучка, намного превосходящих предельный вакуумный ток, вплоть до $\mu^{2}\gamma S/R^{2}$ раз. Более того, с помощью ультрарелятивистских электронных пучков в таких системах оказывается возможным эффективное возбуждение колебаний с длиной волны, намного меньшей поперечных размеров резонатора $\lambda \approx R/\gamma^{2}$ .

\begin{figure}[h!]
	\center{\includegraphics[width=70mm]{dispers_14_3.JPG}}
	%\caption{\label{fig.2.4.1}}
\end{figure}

Другими словами, плазменная СВЧ электроника открывает возможиость создания коротковолновых источников мощного электромагнитного излучения.


Без формул, наверное сразу можно нарисовать дисперсионку резонатора, без пучка.  Как мы видим при добвалении плазмы, появилась ещё одна нижняя ветвь. Точки 1. это работа в режиме усилителя. Пучок (наклонные линии $\omega=ku\pm (\Omega/\gamma)$) усиливает волну бегущую в ту же сторону. Точки 2. это работа в режиме генератора. Пучок усиливает встречную волну.



плазменная СВЧ электроника.
Компрессоры импульсов.


\subsection{Преобразование тепловой энергии в электрическую: МГД-преобразователи, тепловые преобразователи}
\label{14.5} 
[райзер, дополнение, стр. 673]

\begin{figure}[h!]
	\center{\includegraphics[width=70mm]{MGD_generator.JPG}}
	%\caption{\label{fig.2.4.1}}
\end{figure}

Принцип работы МГД генератора представлен на рисунке. Ионизованный газ протекает по каналу со скоростью и и пересекает линии напряженности приложенного постоянного магнитного поля. В движущейся среде индуцируется электрическое поле $Е' =\frac{1}{c} [V\; В]$. Вектор магнитной индукции $В =\mu H$ не отличается от $Н$, поскольку магнитная проницаемость газа $\mu=1$. Направление Е' характеризуется  
координатным равенством $E’_y=-u_x B_z / c $. Индуцированное поле Е' движет электроны вверх. У верхней  
стенки канала накапливается отрицательный пространственный заряд, у нижней — положительный. Если внешняя цепь между электродами, помещенными на этих стенках, разомкнута, заряды накапливаются, пока созданное ими электрическое поле $E_{\inf}$ не уничтожит индуцированное ($E_{\inf}=-E’$). Потенциал нижнего электрода при этом превышает потенциал верхнего на $V_{\inf} = E_{inf} L$. Это напряжение совпадает по величине с ЭДС генератора:
\begin{equation}
	\varepsilon=E’_{y}L=(-u_x /c)(B_z L)
\end{equation}

Если цепь замкнуть через нагрузочное сопротивление и обеспечить достаточно высокую электронную эмиссию с нижнего электрода, например путем его активирования и нагревания, в цепи потечет ток. Реально электроны в газе непрерывным потоком дрейфуют вверх, исходя из нижнего положительного электрода, который в отличие от обычного разряда работает подобно катоду, и внедряясь в верхний. Частичное устранение приэлектродного пространственного заряда, уносимого током, приводит к уменьшению напряжения на электродах $V= Е_{y} L$ по сравнению с $V_{inf}=|\varepsilon|$. В пренебрежении эффектом Холла, в результате которого возникает ток по оси $х \perp Е, В$, плотность тока в газе:
\begin{equation}
	j=\sigma(E+E’)=\sigma(E+\frac{1}{c}[u\;B])
\end{equation}

Ну или в нашей системе координат 
\begin{equation}
	j_y=\sigma(E_y-\frac{[u\;B]}{c}) <0
\end{equation}

Дело ещё обстоит в том, что ток в цепи как раз и зависит от E меж электрожного промежутка, как $i*R=V$ (общий ток через систему с площадью $S$, $i=j*S$), в то время как $V=E*L$, то есть мы получили систему самосогласованных уравнений, которые замкнулись сами на себя. А именно:
\begin{equation}
	i=S \sigma(V/L-uB/c)=S \sigma(iR/L-uB/c)
\end{equation}
отсюда можно выразить $i$ в чистом виде. $i=\frac{u\;B}{c(\frac{R}{L}-\frac{1}{\sigma S})}$

При протекании тока на газ действует тормозящая пондеромоторная сила, в результате чего он и теряет энергию. Отношение мощности, выделяемой в нагрузке, к мощности, отбираемой от газа: 
\begin{equation}
	\frac{iV}{(u/c)(jB)LS}=\frac{V}{\varepsilon}
\end{equation}

т. е. электрический КПД генератора тем выше, чем ближе ситуация к режиму разомкнутой цепи. Но абсолютное значение полезной мощности $iV$ максимально, когда сопротивления нагрузки и плазмы одинаковы и $V=|\varepsilon|/2$


\subsection{Химические реакции в равновесной и неравновесной плазме. \textcolor[rgb]{1,0,0}{Механизмы и кинетика осуществления плазмохимических реакций, роль заряженных и возбужденных частиц. Энергетика химических реакций в электрических разрядах}. Закалка продуктов плазмохимических процессов.}

\label{14.6}

 
[В.Д. Русанов, А.А. Фридман, Физика химически-активной плазмы]

Очевидно, что хотели увеличить скорость некоторых процессов и сократить их количество. Поэтому некоторые реакции хотели провести напрямую в плазме. В рамках классической кинетики при этом экспоненциально ускоряется в соответствии с известным законом Аррениуса $ к \sim exp(-E_{a}/T)$, где $E_a$ — характерная величина энергетического барьера реакции; Т — температура системы. (Понятно, что хим реакция протекает при столкновении двух/трёх частиц, каждая из которых может быть подчинена максвеллу).
 
Резкое повышение скорости процесса может иметь место уже при температурах 3000—5000° С, что значительно выше обычного диапазона температур химических, технологических и  металлургических процессов. Скорость переноса реагентов через зону реакции, очевидно, должна соответствовать повышенной удельной производительности, которая достигается в реакционной зоне при очень высоких температурах, что можно реализовать только в случае газового транспорта реагентов с повышенными скоростями ($5*10^{3}—5*10^{4}$ см/с). 
Очевидно, что подвод к газу столь высокопотенциального тепла не может быть осуществлен с помощью традиционных методов теплопередачи. Сверхвысокие температуры в реакционной зоне должны обеспечиваться подводом энергии электромагнитного поля от внешнего источника, а не тепловой энергии. Этим требованиям отвечает совмещение реакционной зоны с газоразрядной, в которой можно организовать локальный нагрев реагентов до высоких температур за счет активных потерь энергии электромагнитного поля.

Плазмохимические системы в целом можно условно разделить на два больших класса — квазиравновесные и неравновесные.
Температура Т в случае, когда система близка к равновесию (квазиравновесию), является единственной энергетической характеристикой системы, т. е. $Т \approx T_{o} \approx Т_{e} \approx T_{i} \approx  Т_{r} = T_{v}$, где $Т_{o}$ — поступательная температура нейтрального газа, $Т_e$ — температура электронов; $Т_i$ — температура ионов; $Т_r$ — вращательная температура молекул; $T_v$— колебательная температура молекулярного газа. 

Если равновесие в системе не достигнуто, то могут реализоваться различные неравновесные (неизотермические) ситуации, из которых наибольший практический интерес с точки зрения оптимизации характеристик реакционной зоны имеют простейшие системы 

\begin{equation}
T_e \textgreater T_0, T_i, T_v,T_r
\end{equation}

\begin{equation}
T_e \textgreater T_i \textgreater T_0  \approx T_v  \approx T_r
\end{equation}

Неравновесность первого типа представляется наиболее универсальной, при этом концентрация электронов в реакционной зоне оказывается существенно ниже, чем определенная формулой Саха по температуре электронов Те, но на много порядков больше величины пе, определенной по этой же формуле для температуры  нейтральных частиц $Т_0$. Частный случай неравновесной системы, реализуемой во втором случае, связан не только с отрывом температуры электронов, но также с отрывом колебательной температуры  молекул от вращательной и поступательной температуры газа, который может обеспечить реализацию химических процессов через колебательное возбуждение реагирующих молекул. 

При накачке электронным ударом происходит преимущественное заселение нижних колебательных уровней молекул. Заселение высоковозбужденных состояний, проявляющих химическую активность, происходит в основном за счет обмена колебательными квантами. В результате этого процесса с учетом ангармонизма молекул могут возникать распределения по колебательным состояниям, заметно отличающиеся от больцмановского $f(v) \sim ехр(-\hbar \omega \nu / T_v)$, где $\hbar \omega \nu$ - колебательная энергия. Например, для простейшего случая двухатомных молекул учет ангармонизма ($х_e$) приводит к обогащению заселенности высоковозбужденных состояний согласно соотношению Тринора:

\begin{equation}
  f(v) \sim ехр(-\frac{\hbar \omega \nu} {T_v}+x_e\frac{\hbar \omega \nu^{2}}{T_0})
\end{equation}

Доля высоковозбужденных реакционноспособных молекул согласно этому соотношению может быть выше больцмановской на много порядков. Соответственно высокими оказываются и скорости плазмохимических процессов. Это позволяет подавить вредное влияние релаксации и обратных химических реакций, обеспечивая в итоге высокую энергетическую эффективность. 

Подробнее остановимся на неравновесных разрядах. [Д.И. Словецкий, механизмы химических реакций в неравновесной плазме]
Именно на практике термин чаще встречается “неравновесный разряд” в отношении возбуждённых частиц. А именно это распространено в создании мощных газоразрядных лазеров. Понятно, что нам энергетически выгодно заселять только тот уровень, который бы потом излучил. С ростом энергии у нас будут сначала будет возбуждаться поступательная степень свободы, потом вращательная и самая последняя - колебательная. И вот иногда, во время столкновений нейтрала с возбуждённой частицей может возбудиться именно колебательная степень свободы, минуя первые две. тем самым, у нас создаётся неравновесное распределение заселённости по уровням.
В таком случае, вообще, надо начинать заного писать кин уравнение для частиц сорта A на уровне i.
\begin{equation}
\frac{\delta [A(i)] f_{A(i)}}{\delta t}+\frac{F_A}{M_A}\frac{\delta [A(i)] f_{A(i)}}{\delta V_A} +v_A \frac{\delta [A(i)] f_{A(i)}}{\delta r}=\sum_{S,n} I(A,i|S,n) 
\end{equation}

Поскольку классическая химическая кинетика имела дело с сумарными концентрациями чатиц определённого сорта, то и в этом случае можно просуммировать концентрации по всем уровням энергии интегрируя по всем скоростям можно получить
 \begin{equation}
-\frac{d [A] }{dt}=k(C,D|A,B) [A][B]-k(A,B|C,D) [C][D]
\end{equation}
В общем случае коэффициенты реакции могут зависеть от всего на свете, они так же могут иметь пороговый характер. И для описания неравновесных процессов обычно выписываются все уравнения для основных уровней всех веществ, участвующих в реакции и решается на компьютере.

[Неравновесная колебательная кинетика, под редакцией М. Капителли, пункт 4, стр 105]

Самые основные процессы в такой плазме $VT$ (колебательно поступательный), $VV$ (колебательно колебательный) обмен энергии. (много непонятной теории)

\subsection{Методы диагностики химически активной плазмы.}
\label{14.7} 


 [словецкий, стр 39]

\subsection{Взаимодействие плазмы с поверхностью твердых тел. Плазменные технологии (травление, имплантация, упрочнение, нанесение покрытий и пр.)}
\label{14.8}


 [СТЕПАНОВСКИЙ А. С. ИОННАЯ ОБРАБОТКА МАТЕРИАЛОВ ]


В зависимости от параметров ионного или электронного потока в поверхностном слое деталей могут происходить различные процессы. При энергии ионов (уже не инертного газа) $E \approx 10…100 эВ$ на поверхности детали происходит конденсация ионов. Такая обработка используется для осаждения покрытий. Если $E \approx 10^2…10^3 эВ$, то реализуется процесс ионного распыления (травления), который применяется для очистки поверхностей деталей, активирования поверхностного слоя, формирования микрорельефа. При $E > 10^4 эВ$ происходит ионная имплантация, т.е. внедрение ионов имплантируемого вещества в поверхностный слой детали. Таким образом, можно модифицировать поверхностный слой путем его легирования практически любыми элементами. Имплантация приводит к увеличению концентрации дефектов в облучаемом материале, которые в этом случае называют радиационным. При больших дозах облучения поверхностного слоя детали становятся аморфными из-за высокой концентрации радиационных дефектов. Ионная бомбардировка и облучение электронами приводит к нагреву поверхностного слоя, а в ряде случаев наблюдается образование газовых пузырьков и микропор, которые приводят к уменьшению плотности материала


При плазменном травлении обрабатываемый образец помещается непосредственно в область химически активной плазмы, располагаясь на специальном подложкодержателе, и
находится обычно под плавающим потенциалом. Основными частицами, участвующими в процессе плазменного травления и влияющими на него, являются свободные атомы, радикалы, ионы и электроны. Вклад этих частиц в плазменное травление различен: химически активные частицы, т.е. свободные атомы и радикалы, вступают в химическую реакцию с поверхностными атомами материалов и удаляют поверхностные слои в
результате образования летучих продуктов реакции, а электроны и ионы активируют эту реакцию, увеличивая скорость травления. Активирующее воздействие ионов и электронов
определяется энергией, с которой они бомбардируют обрабатываемую поверхность. К плазменному травлению относятся процессы, в которых энергия ионов не превышает 100 эВ.



\newpage
\addcontentsline{toc}{section}{Список литературы}
\bibliographystyle{unsrt}
\bibliography{program}

\end{document}
