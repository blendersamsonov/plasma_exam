\documentclass[10pt, a4paper]{article}
\usepackage[T2A]{fontenc}
\usepackage[english, russian]{babel}

\usepackage{graphicx}
\graphicspath{{./figs/}}

\usepackage{amsmath}
\usepackage{amssymb}
\usepackage{mathtools}
\usepackage{physics}
\usepackage{upgreek}
\usepackage{sectsty}
\usepackage{color,soul}
\usepackage[colorlinks=true,allcolors=blue]{hyperref}
\sectionfont{\centering}

\usepackage{geometry}
 \geometry{
 a4paper,
 margin=25mm,
 }

\title{Билеты к кандидатскому экзамену по физике плазмы}
\date{}

\begin{document}

\newpage
\section{Термодинамика плазмы}
\label{sec.1}

\subsection{Понятие плазмы, квазинейтральность, микрополя, дебаевский радиус, идеальная и неидеальная плазма.}
\label{sec.1.1}

Плазма - это частично или полностью ионизированный газ, в котором меются следующие свойства  - наличие радиуса дебая 
($r_d=\left(\frac{T}{8\pi n e^2}\right)^2$), наличие ленгмюроовских волн (или плазменных волн, их можно рассматривать,
как колебания слоя электронов относительно ионов на частоте $\omega_p^2=\frac{4\pi n e^2}{m}$), а так же затухания Ландау. 

У плазмы вводят т.н. параметр неидеальности, который есть $g = \frac{1}{n * r_d^3}$(Kroll, eq. 1.3.1). Когда 
$g \ll 1$ говорят, что плазма идеальна, т.к. в этом случае её свойства, как газа, схожи с идеальным газом.





~\cite{kotelnikov}. Ссылка на уравнение~\eqref{eq.Euler}, Рис.~\ref{fig.1.2.1}.


\begin{equation}
    \label{eq.Euler}
    e^{i \pi} + 1 = 0
\end{equation}

\subsection{Условие термодинамического равновесия, термическая ионизация, формула Саха, корональное равновесие, снижение потенциала ионизации.}
\label{sec.1.2}

Картинка

\begin{figure}[h!]
    \center{\includegraphics[width=85mm]{test.pdf}}
    \caption{\label{fig.1.2.1} Подпись.}
\end{figure}

\subsection{Вырождение плазмы, статистика Больцмана и Ферми—Дирака, модель Томаса—Ферми.}


\section{Элементарные процессы}
\label{sec.2}

\subsection{Столкновения заряженных частиц, дальнодействие.}
\label{sec.2.1}

Из всех сил взаимодействия между атомными частицами медленнее вceгo спадают с расстоянием (как $1/r^{2}$)  кулоновские силы. Они обладают наибольшим дальнодействием.
За время пролёта t мимо иона, электрон отклоняется на угол $\theta$ . Его можно оценить как отношение полученной поперечной скорости к изначальной скорости:

Основную роль в рассеянии играют столкновения с большим прицельным параметром $\rho$ (рассеяние на малые углы),реализуются при  $\rho > r_{0}$- Кулоновского радиуса (радиус при котором кин. энергия электрона равна потенциальной $ mv^{2}/2 = e^{2}/ r_{0} $,$ r_{0}=e^{2}/mv^{2} $). С другой стороны, потенциал иона спадает $\sim exp(-r/d)/r $. То есть основной вклад вносят столкновения с прицельным параметром от  $r_{0}$   до $d$ (радиус дебая).
Поэтому полное сечение для кулоновских рассеяний $\sigma= \pi*r_{0}^{2} \int_{r_{0}}^{d} r_{0} d\rho/\rho =\pi*r_{0}^{2}*ln{d/r_{0}}$

$\ln{d/r_{0}}$ - Кулоновский логарифм
Столкновения атомных частиц могут иметь упругий и неупругий характер. При упругом соударении меняются направления движения партнеров, происходит обмен импульсом и кинетической энергией, но внутренние энергии и состояния частиц остаются неизменными.

\section{Колебания и волны в плазме}
\label{sec.7}

\subsection{Столкновения заряженных частиц, дальнодействие.}
\label{sec.2.1}
В изотропной плазме легко записат дисперсионное уравнение:

\begin{equation}
    \label{eq.Disp1}
    \left(1-\frac{\omega^2}{c^2 k^2} \varepsilon_{\perp}\right) E_{\perp} + \varepsilon_{\parallel} * E_{\parallel}= 0
\end{equation}

Отсюда, при условии $\varepsilon_{\parallel} = 0$ получаем первый тип волн - продольные(ленгмюровские/электростатические) 
волны. Затухание Ландау(безстолкновительное затухание) относится именно к таким волнам. Их дисперсионка имеет следующий
вид:

\begin{equation}
    \label{eq.Disp2}
    \omega=\omega_p + 3/2 (k v_{t})^2
\end{equation}
волны с частотой сильно отличной от плазменное затухают(т.е. это плазменное колебание).

В случае $1 - \frac{\omega^2}{c^2 k^2} \varepsilon_{\parallel}=0$ получаем поперечные волны(почти как в вакууме). Для них
дисперсионка:

\begin{equation}
    \label{eq.Disp3}
    \omega^2=\omega_p^2 + (c k)^2
\end{equation}
Легко видеть, что волны с частотой $\omega < \omega_p$ не распространяются.

Если еще учесть ионы:

\begin{equation}
    \label{eq.Disp4}
    \varepsilon=1+\frac{\omega_{pe}^2}{k^2 v_{Te}^2} - \frac{\omega_{pi}^2}{\omega^2}-3\frac{\omega_{pi}^2 k^2 v_{Ti}^2}{\omega^4}
\end{equation}
В случае низких частот(где влияние ионов существенно) можно принебреч 4ым слашаемым, тогда дисперсионка примет следующий вид:

\begin{equation}
    \label{eq.Disp5}
    \omega=\frac{k c_s}{\sqrt{1 + k^2 r_d^2}}
\end{equation}
при малых $k$ имеем $\omega=k c_s$, где $c_s=\frac{T_i + T_e}{M}$ (ионный звук).

Магнитоактивная плазма
\newpage
\addcontentsline{toc}{section}{Список литературы}
\bibliographystyle{unsrt}
\bibliography{program}

\end{document}
